Prognozowanie:
\begin{enumerate}[a)]
\item Racjonalna - logiczny proces przebiegający od przesłanek do konkluzji/wniosków.
\item Nieracjonalne - proces przebiegający bez przesłanek (nie są znane lub są znikome) lub od przesłanek nie mających związku ze zjawiskiem do konkluzji.
\item Naukowe - gdy w procesie wnioskowania korzystamy z twierdzeń, reguł, aksjomatów, faktów nauki.
\end{enumerate}
Prognoza
\begin{enumerate}[a)]
\item sąd, którego prawdziwość jest zdarzeniem losowym, przy czym prawdopodobieństwo tego zdarzenia jest nie mniejsze niż  z góry ustalona, bliska jedności liczba zwana wiarygodnością.
\item to konkretny wynik wnioskowanie na podstawie znajomości modelu opisującego pewien odcinek sfery zjawisk
\item to sąd charakteryzujący się własnościami:
\begin{itemize}
\item jest sformułowany z wykorzystaniem dorobku nauki
\item odnosi się do przyszłości
\item jest weryfikowany empirycznie
\item jest niepewny, ale możliwy do sprawdzenia
\end{itemize}
\end{enumerate}
Prognozy makroekonomiczne - są budowane przez specjalistyczne organizacje/instytucje. Przewiduję one między innymi zmienność stóp procentowych, stóp inflacji, służą do analizy popytu rynku w w skali globalnej, mają szeroki wpływ/zakres działania.\\
Prognozy mikroekonomiczne - budowane w firmach/przedsiębiorstwach, które zajmują się badaniem rynku w skali llokalnej o wąskim marginesie wpływu.

Podział prognoz:
\begin{itemize}
\item krótkoterminowa - dotyczy odpowiednio krótkiego czasu, w którym można zaobserwować zmiany ilościowe
\item średnioterminowa - dotyczy odpowiednio średniego czasu, w którym można zaobserwować zmiany ilościowe
\item długoterminowa - dotyczy odpowiednio długiego czasu, w którym można zaobserwować zmiany ilościowe
\end{itemize}
Metody prognozowania:
\begin{itemize}
\item matematyczno-statystyczne
\item ekonometryczne
\item jednorównaniowe i wielorównaniowe
\item ankietowe
\item intuicyjne i refleksyjne
\end{itemize}
Etapy prognozowania
\begin{itemize}
\item sformułowanie zadania prognostycznego
\item podanie przesłanek prognostycznych
\item wybór metody prognozowania
\item wyznaczenie prognozy
\item ocena trafności prognozy (analiza błędów; weryfikacja)
\end{itemize}
Błędy:
\begin{enumerate}[a)]
\item ex post - podaje wartość odchylenia rzeczywistych wielkości od prognozowanych
\item ex ante - wyraża spodziewaną wielkość odchyleń prognoz od realizacji zmiennej prognozowanej
\end{enumerate}
Definicje\\
Szereg czasowy
\begin{itemize}
\item realizacja procesu stochastycznego, którego dziedziną jest czas
\item ciąg informacji uporządkowanych w czasie (często zakłada się dokładny, równomierny krok czasowy)
\end{itemize}
Trend - składnik szeregu czasowego (monotoniczny - wyraźnie) będący funkcją, która opisuje zachowanie szeregu czasowego, tzw. ogólny kierunek, ogólna tendencja rozwojowa, względnie trwały.
\begin{multicols}{2}
\centering W modelu addytywnym
\begin{gather*}
Y=M+T+C+S+K+I+\xi
\end{gather*}
\centering W modelu multiplikatywnym
\begin{gather*}
Y=MTCSKI\xi
\end{gather*}
\end{multicols}\noindent
$ M $ - stała wartość, tzw. przeciętny poziom zjawiska\\
$ T $ - trend\\
$ C $ - cykle długookresowe, czyli wahanie tzw. regularne (o cyklu długim)\\
$ S $ - wahania sezonowa\\
$ K $ - wahania krótkoterminowe\\
$ I $ - "jednorazowe" zmiany\\
$ \xi  $ - składnik losowy

Modele adaptacyjne:
\begin{itemize}
\item modele średniej
\item modele średniej ruchomej
\item modele naiwne
\item wygładzanie wykładnicze
\end{itemize}
\begin{gather*}
y_k=\frac{1}{k}\sum_{l=t-k}^{t-1}y_l\\
y_t=\sum_{i=t-k}^{t-1}y_iw_{i-(t-k)+1}\qquad \sum w_i=1
\end{gather*}
Metody naiwne:
\begin{itemize}
\item oparta na błędzie losowym\\
$ \dot y_t=y_n\\
\dot y_t=y_t\\
\dot y_t=y_{t-1} $
\item dla szeregu z trendem\\
$ \dot y_t=y_n+\left(y_n-y_{n-1}\right) $
\item dla szeregu z wahaniami (np. sezonowymi)\\
$ \dot y_t=y_{n+1-m}\\
\dot y_t=y_{n-m} $
\end{itemize}