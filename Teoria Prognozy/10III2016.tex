\textbf{Uwaga!}\\
Trend może być rosnący, malejący, boczny, monotoniczny przedziałami.\\
Postać trendu
\begin{itemize}
\item liniowa\\
$ y_t=at+b $
\item wykładnicza\\
$ y_t=ba^t $
\item ekspotencjalna\\
$ be^t $
\item logarytmiczna\\
$ y_t=b\cdot \ln t $
\item potęgowa\\
$ y_t=bt^a $
\end{itemize}
W celu dopasowania funkcji trendu (do danych rzeczywistych) należy określić parametry (tzw. parametry dopasowania, parametry struktury) stochastycznej).
\begin{itemize}
\item odchylenie standardowe składnika lodowego - ten parametr informuje, o ile wartości empiryczne różnią się od wartości teoretycznych, wyznaczonych na podstawie trendu
\item współczynnik zmienności resztowej - określa jaka część średniej arytmetycznej badanej zmiennej stanowi odchylenie standardowe składnika resztowego
\item współczynnik zbieżności - informuje on jaka część zmienności zmiennej objaśnianej nie została objaśniona przez funkcję trendu, przyjmuje wartości od 0 do 1. Im wartość współczynnika jest bliższa zeru, tym lepiej
\item współczynnik determinacji określa, jaka część zmienności zmiennej objaśnianej została wyjaśniona przez funkcję trendu, przyjmuje wartości od 0 do 1. Im bliższy 1 tym lepiej
\end{itemize}
\section{Model wygładzania wykładniczego (tzw. model Browna).}
Ten model stosowany jest, gdy w szeregu czasowym występuje prawie stały poziom zmiennej prognozowanej oraz wahań przypadkowych (wahania małe).\\
(Założenie: Trend boczny, mało zaburzeń przypadkowych, wahania małe)
\begin{align*}
&y_{t-1}=F_{t-2}=\frac{1}{k}\left[y_{t-2}+y_{t-3}+\dots+y_{t-k+1}\right]\\
&\dot y_t=F_{t-1}=\frac{1}{k}\left[y_{t-1}+y_{t-2}+\dots+y_{t-k}\right]\\
&\dot y_t=F_{t-1}=\frac{1}{k}\left[y_{t-1}-y_{t-k-1}\right]+\dot y_{t-1}\\
&k\neq 0,\; k\in \mathbb N\\
&t-k-1>0\Rightarrow t>k+1\\
&\frac{1}{k}=\alpha,\;\alpha=\frac{1}{2},\frac{1}{3},\frac{1}{4},\dots,\frac{1}{N}
\end{align*}
Poszerzmy zakres $ \alpha $ do $ (0,1) $
\begin{align*}
&\dot y_t=\frac{1}{k}y_{t-1}+\left(1-\frac{1}{k}\right)\dot y_{t-1}\\
&y_t=\alpha\left(y_{t-1}\right)+\left(1-\alpha\right)\dot y_{t-1}
\end{align*}
Następnie $ \dot y_{t-1} $ zastępujemy przez $ \alpha y_{t-2}+\left(1-\alpha\right)\dot y_{t-2}$
\begin{gather*}
\dot y=\alpha y_{t-1}+\alpha\left(1-\alpha\right)y_{t-2}+\left(1-\alpha\right)^2\dot y_{t-2}
\end{gather*}
Krok następny
\begin{gather*}
\dot y=\alpha y_{t-1}+\alpha\left(1-\alpha\right)y_{t-2}+\alpha\left(1-\alpha\right)^2y_{t-3}+\left(1-\alpha\right)^3\dot y_{t-3}
\end{gather*}
\textbf{Uwaga!}\\
Najczęściej przyjmuje się wartość początkową zmiennej prognozowanej lub średnią arytmetyczną rzeczywistych wartości zmiennej.