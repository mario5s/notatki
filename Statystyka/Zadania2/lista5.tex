\chapter*{Lista 5}
\addcontentsline{toc}{chapter}{Lista 5}


\subsection*{Zadanie 52}
\addcontentsline{toc}{section}{Zadanie 52}
Wyznaczyć płaszczyznę (podać jej równanie ogólne) na której skoncentrowany jest normalny wektor $ X=(X_1,X_2,X_3) $ o macierzy kowariancji $ R $ i o wektorze średnim $ m $, jeśli
\begin{gather*}
m=(1,2,1),\qquad
R=
\begin{bmatrix}
	5  & 4 & -2 \\
	4  & 5 & 2  \\
	-2 & 2 & 8
\end{bmatrix}.
\end{gather*}
Rozwiązanie:
\begin{itemize}
\item Równanie ogólne płaszczyzny:
\begin{gather*}
Ax+By+Cz+D=0
\end{gather*}
\end{itemize}
Sprawdzamy rząd macierzy $ R $, bo jeśli wynosi 3 lub 1 to nie ma czego wyznaczać.
\begin{gather*}
\det(R)=
\det\left(
\begin{bmatrix}
	5  & 4 & -2 \\
	4  & 5 & 2  \\
	-2 & 2 & 8
\end{bmatrix}
\right)=0
\end{gather*}
Zatem rząd tej macierzy to maksymalnie 2. Weźmy pod uwagę minor główny rozmiaru 2 i policzmy jego wyznacznik.
\begin{gather*}
\det \left(
\begin{bmatrix}
	5 & 4 \\
	4 & 5
\end{bmatrix}
\right)=
9.
\end{gather*}
Oznacza to, ze $ rz(R)=2 $. Aby wyznaczyć szukaną płaszczyznę należy wybrać 2 liniowo niezależne wektory spośród wierszy macierzy $ R $.
\begin{gather*}
v_1=\left(5,4,-2\right)\\
v_2=\left(4,5,2\right).
\end{gather*}
Wyznaczamy wektor prostopadły do wybranych wektorów $ v_1 $ i $ v_2 $ poprzez iloczyn wektorowy.
\begin{gather*}
v_1\times v_2=\left(18, -18, 9\right)\cong \left(2, -2, 1\right)
\end{gather*}
Jest to wektor normalny szukanej przez nas płaszczyzny. Ponadto wiemy, że punkt $ m $ należy do tej płaszczyzny, więc ostatecznie równanie naszej płaszczyzny wygląda następująco
\begin{gather*}
2 (x - 1) - 2 (y - 2) + 1 (z - 1)=0\\
2 x-2 y+z+1=0
\end{gather*}


\subsection*{Zadanie 53}
\addcontentsline{toc}{section}{Zadanie 53}
Wektor losowy $ X=(X_1,X_2,X_3) $ ma rozkład normalny o macierzy kowariancji $ R $ i o wektorze średnim $ m $, gdzie
\begin{gather*}
m=(1,2,1),\qquad
R=\begin{bmatrix}
	5  & 4 & -2 \\
	4  & 5 & 2  \\
	-2 & 2 & 8
\end{bmatrix}.
\end{gather*}
Podać równanie ogólne płaszczyzny, na której jest skoncentrowany wektor losowy $ Y=AX $, jeśli
\begin{gather*}
A=
\begin{bmatrix}
	1 & 0  & 1 \\
	1 & -1 & 1 \\
	1 & 1  & 0
\end{bmatrix}.
\end{gather*}

Rozwiązanie:
\begin{itemize}
\item Jeżeli $ X\sim \mathcal N(m,R) $ to $ Y\sim \mathcal N(Am,ARA^T) $
\end{itemize}
Nasze zadanie zaczyna się od wyznaczenia parametrów rozkładu zmiennej losowej $ Y $.
\begin{gather*}
m'=Am=
\begin{bmatrix}
	1 & 0  & 1 \\
	1 & -1 & 1 \\
	1 & 1  & 0
\end{bmatrix}
\cdot
\begin{bmatrix}
	1 \\
	2 \\
	1
\end{bmatrix}
=
\begin{bmatrix}
	2 \\
	0 \\
	3
\end{bmatrix}\\
R'=ARA^T=
\begin{bmatrix}
	1 & 0  & 1 \\
	1 & -1 & 1 \\
	1 & 1  & 0
\end{bmatrix}\cdot
\begin{bmatrix}
	5  & 4 & -2 \\
	4  & 5 & 2  \\
	-2 & 2 & 8
\end{bmatrix}\cdot
\begin{bmatrix}
	1 & 1  & 1 \\
	0 & -1 & 1 \\
	1 & 1  & 0
\end{bmatrix}
=\\=
\begin{bmatrix}
	3  & 6 & 6 \\
	-1 & 1 & 4 \\
	9  & 9 & 0
\end{bmatrix}\cdot
\begin{bmatrix}
	1 & 1  & 1 \\
	0 & -1 & 1 \\
	1 & 1  & 0
\end{bmatrix}=
\begin{bmatrix}
	9 & 3 & 9  \\
	3 & 2 & 0  \\
	9 & 0 & 18
\end{bmatrix}
\end{gather*}
Podobnie jak poprzednio sprawdzamy rząd macierzy $ R' $
\begin{gather*}
\det \left(
\begin{bmatrix}
	9 & 3 & 9  \\
	3 & 2 & 0  \\
	9 & 0 & 18
\end{bmatrix}
\right)=0
\end{gather*}
Sprawdźmy wyznacznik imnora głównego
\begin{gather*}
\det \left(
\begin{bmatrix}
	9 & 3 \\
	3 & 2
\end{bmatrix}\right)=9
\end{gather*}
Wybieramy jako wektory dwa pierwsze wiersze macierzy $ R' $
\begin{gather*}
v_1=(9,3,9)\\
v_2=(3,2,0)
\end{gather*}
Wyznaczamy wektor prostopadły do wybranych wektorów $ v_1 $ i $ v_2 $ poprzez iloczyn wektorowy.
\begin{gather*}
v_1\times v_2=(-18,27,9)\cong(-2,3,1).
\end{gather*}
Podobnie jak poprzednio, wyznaczamy równanie płaszczyzny
\begin{gather*}
-2 (x-2)+3 y+z-3=0\\
-2 x+3 y+z+1=0
\end{gather*}


\subsection*{Zadanie 54}
\addcontentsline{toc}{section}{Zadanie 54}
Wektor losowy $ X=(X_1,X_2,X_3) $ ma rozkład normalny o macierzy kowariancji $ R $ i o wektorze średnim $ m $, gdzie
\begin{gather*}
m=(1,2,1),\qquad
R=\begin{bmatrix}
	5  & 4 & -2 \\
	4  & 5 & 2  \\
	-2 & 2 & 8
\end{bmatrix}.
\end{gather*}
Podać równanie krawędziowe prostej, na której jest skoncentrowany wektor losowy $ Y=AX $, jeśli
\begin{gather*}
A=
\begin{bmatrix}
	1  & 2  & 1  \\
	2  & 4  & 2  \\
	-1 & -2 & -1
\end{bmatrix}.
\end{gather*}
\newpage
Rozwiązanie:\\
Wyznaczmy parametry rozważanego rozkładu.
\begin{gather*}
m'=Am=
\begin{bmatrix}
	1  & 2  & 1  \\
	2  & 4  & 2  \\
	-1 & -2 & -1
\end{bmatrix}\cdot
\begin{bmatrix}
	1 \\
	2 \\
	1
\end{bmatrix}=
\begin{bmatrix}
	6  \\
	12 \\
	-6
\end{bmatrix}\\
R'=ARA^T=
\begin{bmatrix}
	1  & 2  & 1  \\
	2  & 4  & 2  \\
	-1 & -2 & -1
\end{bmatrix}\cdot
\begin{bmatrix}
	5  & 4 & -2 \\
	4  & 5 & 2  \\
	-2 & 2 & 8
\end{bmatrix}\cdot
\begin{bmatrix}
	1 & 2 & -1 \\
	2 & 4 & -2 \\
	1 & 2 & -1
\end{bmatrix}
=\\=
\begin{bmatrix}
	11  & 16  & 10  \\
	22  & 32  & 20  \\
	-11 & -16 & -10
\end{bmatrix}\cdot
\begin{bmatrix}
 	1 & 2 & -1 \\
 	2 & 4 & -2 \\
 	1 & 2 & -1
\end{bmatrix}=
\begin{bmatrix}
	53  & 106  & -53  \\
	106 & 212  & -106 \\
	-53 & -106 & 53
\end{bmatrix} 
\end{gather*}
Wiemy, że do szukanej prostej należy punkt $ m' $ oraz, że wyznacza ją wektor uzyskany z macierzy $ R' $, której rząd wynosi 1.
\begin{gather*}
l:
\left \{
\begin{array}{l}
x=53t+6\\
y=106t+12\\
z=-53t-6
\end{array}
\right .,\quad t\in \mathbb R \\
l:
\left \{
\begin{array}{l}
x=t+6\\
y=2t+12\\
z=-t-6
\end{array}
\right .,\quad t\in \mathbb R \\
l:
\left \{
\begin{array}{l}
x=t\\
y=2t\\
z=-t
\end{array}
\right .,\quad t\in \mathbb R 
\end{gather*}
Do równania krawędziowego potrzebne są dwie płaszczyzny zawierające w sobie prostą $ l $ oraz mające różne wektory normalne. Z powyższego układu równań łatwo wyodrębnić dwa równania:
\begin{align*}
&y+2z=0\\
&x+z=0
\end{align*}
ostateczna postać
\begin{gather*}
\lambda_1(y+2z)=\lambda_2(x+z),\qquad\lambda_1,\lambda_2\in \mathbb R 
\end{gather*}



\subsection*{Zadanie 57}
\addcontentsline{toc}{section}{Zadanie 57}
Wykazać, że składowe normalnego wektora losowego $ X=(X_1,\dots,X_n) $ są niezależne wtedy i tylko wtedy, gdy są nieskorelowane. Podać przykład wektora losowego $ (X,Y) $, którego rozkłady brzegowe są normalneg, $ X,Y $ są nieskorelowane, a rozkład $ (X,Y) $ nie jest normalny.

Rozwiązanie:\\
Implikacja w jedną stronę jest oczywista, ponieważ niezależne zmienne losowe są ze sobą nieskorelowane. Zajmijmy się drugą stroną równoważności. $ X\sim\mathcal N(m,R) $.
\begin{gather*}
\forall_{i,j}\;\cov(X_i,X_j)=0\Rightarrow
\forall_{i,j}\;\rho(X_i,X_j)=0
\end{gather*}
Powyższe zależności implikują fakt, iż macierz $ R $ jest diagonalna, czyli funkcja gęstości zmiennej losowej $ X $ ma postać
\begin{gather*}
f_X(t)=
(2\pi)^{-\frac{n}{2}}|R|^{-\frac{1}{2}}
\exp\left(-\frac{1}{2}(t-m)^TR^{-1}(t-m)\right)
=\\=
(2\pi)^{-\frac{n}{2}}|R|^{-\frac{1}{2}}
\exp\left(-\frac{1}{2}\sum_{k=1}^{n}\left((t_k-m_k)^2\cdot \frac{1}{\Var(X_k)}\right)\right)
=\\=
(2\pi)^{-\frac{n}{2}}|R|^{-\frac{1}{2}}
\prod_{k=1}^{n}\exp\left(-\frac{1}{2}\left(\frac{(t_k-m_k)^2}{\Var(X_k)}\right)\right)
=\\=
\prod_{k=1}^{n}\frac{1}{\sqrt{2\pi\Var(X_k)}}\exp\left(-\frac{(t_k-m_k)^2}{2\Var(X_k)}\right)=\prod_{k=1}^{n}f_{X_k}(x_k)
\end{gather*}
Ostatecznie otrzymaliśmy, że jeżeli rozkład spełnia podane warunki to poszczególne jego składowe są niezależne od siebie.