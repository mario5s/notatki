\chapter*{Lista 7}
\addcontentsline{toc}{chapter}{Lista 7}


\subsection*{Zadanie 75}
\addcontentsline{toc}{section}{Zadanie 75}
Niech $ X=(X_1,\dots,X_n),\,n\ge 1 $ będzie próbą z populacji, w której cecha ma $ X $ ma rozkład gamma $ G(a,p) $. Wyznaczyć rozkład statystyki $ \overline X $.

Rozwiązanie:
\begin{itemize}
\item Gęstość rozkładu gamma
\begin{gather*}
f(x)=\frac{1}{\Gamma(p)}x^{p-1}a^p\exp\left(-ax\right)
\end{gather*}
\item Funkcja charakterystyczna
\begin{gather*}
\varphi_x(t)=\left(1 - \frac{it}{a}\right)^{-p}
\end{gather*}
\end{itemize}
Policzmy rozkład przy użyciu funkcji charakterystycznej
\begin{align*}
&\varphi_{\frac{1}{n}\sum_{k=1}^{n}X_k}(t)
=\\=&
\prod_{k=1}^{n}\varphi_{\frac{X_k}{n}}(t)
=\\=&
\left(\varphi_{\frac{X_k}{n}}(t)\right)^n
=\\=&
\left(\varphi_{{X_k}}\left(\frac{t}{n}\right)\right)^n
=\\=&
\left(\left(1 - \frac{it}{an}\right)^{-p}\right)^n
=\\=&
\left(1 - \frac{it}{an}\right)^{-np}
\end{align*}
Statystyka ta ma rozkład $ G(an,ap) $.


\subsection*{Zadanie 79}
\addcontentsline{toc}{section}{Zadanie 79}
Niech $ X=(X_1,\dots,X_n) $ będzie próbą z rozkładu o ciągłej dystrybuancie $ F $. Wykazać, że zmienna losowa $ F(X_{(k)}) $ ma rozkład beta $ B(k,n-k+1) $.

Rozwiązanie:
\begin{itemize}
\item Gęstość rozkładu beta
\begin{gather*}
f(x)=\frac{1}{B (p,q)}x^{p-1}(1-x)^{q-1}=
f(x)=\frac{\Gamma(p)\Gamma(q)}{\Gamma(p+q)}x^{p-1}(1-x)^{q-1}
\end{gather*}
\end{itemize}
Spróbujmy
\begin{align*}
&P\left(F\left(X_{(k)}\right)\le t\right)
=\\=&
\sum_{l=1}^{n}P\left(F\left(X_{(k)}\right)\le t|X_{(k)}=X_l\right)P\left(X_{(k)}=X_l\right)
=\\=&
\sum_{l=1}^{n}P\left(F\left(X_l\right)\le t|X_{(k)}=X_l\right)P\left(X_{(k)}=X_l\right)
=\\=&
\sum_{l=1}^{n}\frac{1}{\binom{n}{k}}P\left(F(X_l)\le t\right)P\left(F(X)\le t\right)^{k-1}P\left(F(X)>t\right)^{n-k}
=\\=&
\frac{n}{\binom{n}{k}}t^{k-1}\left(1-t\right)^{n-k}
=\\=&
n\cdot \frac{k!(n-k)!}{n!}t^k\left(1-t\right)^{n-k}
%\sum_{l=1}^{n}P\left(F\left(X_{(k)}\right)\le t|X_{(k)}=X_l\right)P\left(X_{(k)}=X_l\right)
%=\\=&
%\sum_{l=1}^{n}\frac{P\left(F\left(X_{(k)}\right)\le t,X_{(k)}=X_l\right)}{P\left(X_{(k)}=X_l\right)}P\left(X_{(k)}=X_l\right)
%=\\=&
%\sum_{l=1}^{n}\frac{P\left(F\left(X_l\right)\le t,X_{(k)}=X_l\right)}{P\left(X_{(k)}=X_l\right)}P\left(X_{(k)}=X_l\right)
%=\\=&
%\sum_{l=1}^{n}\frac{P\left(F\left(X_l\right)\le t\right )P\left (X_{(k)}=X_l\right)}{P\left(X_{(k)}=X_l\right)}P\left(X_{(k)}=X_l\right)
%=\\=&
%\sum_{l=1}^{n}P\left(F\left(X_l\right)\le t\right )P\left(X_{(k)}=X_l\right)
\end{align*}
Poddaję się


%\subsection*{Zadanie 80}
%\addcontentsline{toc}{section}{Zadanie 80}
%Niech $ \mathbf X=(X_1,\dots,X_n) $ będzie próbą z rozkładu Poissona z parametrem $ \lambda>0 $. Korzystając z definicji, udowodnić, że $ T(X)=\sum_{i=1}^{n}X_i $ jest statystyką dostateczną dla parametru $ \lambda>0 $.
%
%Rozwiązanie:
%\begin{itemize}
%\item 
%\begin{gather*}
%P(X=k)=\frac{\lambda^k}{k!}e^{-\lambda}
%\end{gather*}
%\end{itemize}
%Należy pokazać, że prawdopodobieństwo warunkowe jest niezależne od parametru $ \lambda $. Rozkłąd statystyki $ T $ wynosi (korzystając z funkcji charakterystycznej)
%\begin{gather*}
%\varphi_{T(\mathbf X)}(t)=
%\varphi_{\sum_{i=1}^{n}X_i}(t)=
%\left(\varphi_{X_1}(t)\right)^n=
%\left(e^{\lambda(e^{it} - 1)}\right)^n=
%e^{n\lambda(e^{it} - 1)}
%\end{gather*}
%Czyli $ T(\mathbf X) $ ma rozkład Poissona z parametrem $ n\lambda $.
%\begin{align*}
%&P\left(\mathbf X=x|T(\mathbf X)\right)
%=\\=
%&\frac{P\left(\mathbf X=x,T(\mathbf X)=t\right)}{P\left(T(\mathbf X)=t\right)}
%\end{align*}