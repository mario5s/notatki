\chapter*{Lista 8}
\addcontentsline{toc}{chapter}{Lista 8}


\subsection*{Zadanie 93}
\addcontentsline{toc}{section}{Zadanie 93}
Sprawdzić, czy
\begin{enumerate}
\item Rozkłady beta tworzą dwuparametrową rodzinę wykładniczą;
\item Rodzina rozkładów Rayleigha z paramterem $ \sigma>0 $ należy do rodziny wykładniczej;
\item Dwuparametrowe rozkłady normalne tworzą pięcioparametrową rodzinę wykładniczą;
\item Rodzina rozkładów normlanych $ \mathcal N(m,\sigma^2) $ należy do rodziny wykładniczej;
\item Rodzina rozkładów o gęstościach
\begin{gather*}
f_\theta (x)=\frac{2(x+\theta)}{1+2\theta},\qquad
x\in(0,1),\qquad\theta \in \Theta=(0,\infty )
\end{gather*}
\end{enumerate}
należy do rodziny wykładniczej.

Rozwiązanie:
\begin{enumerate}
\item 
\begin{align*}
&f_B(x)=\frac{1}{B (p,q)}x^{p-1}(1-x)^{q-1}
=\\=&
\frac{1}{B (p,q)}\exp\bigl(\ln x\cdot (p-1)+\ln (1-x)\cdot (q-1)\bigr)
=\\=&
\frac{1}{B (p,q)}\exp\bigl(\ln x\cdot p+\ln (1-x)\cdot q-\ln x-\ln (1-x)\bigr)
=\\=&
\frac{1}{B (p,q)}\exp\bigl(\ln x\cdot p+\ln (1-x)\cdot q\bigr)\frac{1}{x(1-x)}
\end{align*}
Wypisać funkcje składowe
\begin{align*}
&C(\theta)=C(p,q)=\frac{1}{B(p,q)}
&&
h(x)=\frac{1}{x(1-x)}\\
&T_1(x)=\ln x
&&
T_2(x)=\ln(1-x)\\
&Q_1(\theta)=Q_1(p,q)=p
&&
Q_2(\theta)=Q_1(p,q)=q
\end{align*}
\item Rayleigh...
\item Dwuwymiarowy rozkład normalny
\begin{align*}
&f_N(x,y)=
\frac{1}{2 \pi  \sqrt{\sigma _{1,1} \sigma_{2,2}-\sigma_{1,2}^2}}
\end{align*}
\end{enumerate}