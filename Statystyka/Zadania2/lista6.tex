\chapter*{Lista 6}
\addcontentsline{toc}{chapter}{Lista 6}


\subsection*{Zadanie 61}
\addcontentsline{toc}{section}{Zadanie 61}
Wektor losowy $ W=(X,Y,Z) $ ma rozkład normalny o gęstości
\begin{gather*}
f(x,y,z)=C\exp\left[-\frac{1}{2}\left(2x^2+y^2+3z^2-2xy-2yz+4xz\right)\right]
\end{gather*}
Wyznaczyć $ C $ oraz macierz kowariancji tego wektora losowego oraz wyznaczyć gęstość wektora losowego $ Y=AW $, jeśli
\begin{gather*}
A=\begin{bmatrix}
	1 & 0  & 1 \\
	1 & -1 & 1 \\
	1 & 1  & 0
\end{bmatrix}.
\end{gather*}

Rozwiązanie:
Aby wyznaczyć stałą $ C $ trzeba policzyć całkę z zaprezentowanej gęstości i przyrównać ją do jedynki.
\begin{align*}
&\iiint\limits_{\mathbb R ^3}\exp\left[-\frac{1}{2}\left(2x^2+y^2+3z^2-2xy-2yz+4xz\right)\right]\,dxdydz
=\\=&
\iiint\limits_{\mathbb R ^3}\exp\left[-\frac{1}{2}\left((y-x-z)^2+(x+z)^2+z^2\right)\right]\,dxdydz
=\\=&
2\sqrt 2\pi^\frac{3}{2}\int\limits_{\mathbb R }\frac{1}{\sqrt{2\pi}}\exp\left[-\frac{1}{2}z^2\right]
\int\limits_{\mathbb R }\frac{1}{\sqrt{2\pi}}\exp\left[-\frac{1}{2}\left(x+z\right)^2\right]\dots\\\dots&
\int\limits_{\mathbb R }\frac{1}{\sqrt{2\pi}}\exp\left[-\frac{1}{2}\left(y-x-z\right)^2\right]dydxdz
=\\=&
2\sqrt 2\pi^\frac{3}{2}\int\limits_{\mathbb R }\frac{1}{\sqrt{2\pi}}\exp\left[-\frac{1}{2}z^2\right]
\int\limits_{\mathbb R }\frac{1}{\sqrt{2\pi}}\exp\left[-\frac{1}{2}\left(x+z\right)^2\right]dxdz
=\\=&
2\sqrt 2\pi^\frac{3}{2}\int\limits_{\mathbb R }\frac{1}{\sqrt{2\pi}}\exp\left[-\frac{1}{2}z^2\right]dz
=\\=&
2\sqrt 2\pi^\frac{3}{2}
\end{align*}
Ponieważ
\begin{gather*}
\iiint\limits_{\mathbb R ^3}f(x,y,z)\,dxdydz=
2\sqrt 2\pi^\frac{3}{2}\\
C=\frac{1}{2\sqrt 2\pi^\frac{3}{2}}
\end{gather*}
W celu znalezienia macierzy kowariancji przeanalizujmy wykładnik eksponenty
\begin{gather*}
-\frac{1}{2}(w-\mu)^T\Sigma^{-1}(w-\mu)=
-\frac{1}{2}\left(2x^2+y^2+3z^2-2xy-2yz+4xz\right)\\
(w-\mu)^T\Sigma^{-1}(w-\mu)=
2x^2+y^2+3z^2-2xy-2yz+4xz
\end{gather*}
Oznaczmy
\begin{gather*}
\Sigma^{-1}=
\begin{bmatrix}
	\rho _{1,1} & \rho _{1,2} & \rho _{1,3} \\
	\rho _{1,2} & \rho _{2,2} & \rho _{2,3} \\
	\rho _{1,3} & \rho _{2,3} & \rho _{3,3}
\end{bmatrix},\qquad
\mu=\begin{bmatrix}
\mu_1\\
\mu_2\\
\mu_3
\end{bmatrix}
\end{gather*}
Przyjmując te oznaczenia i rozpisując zaprezentowaną zależność otrzymujemy
\begin{align*}
&(x-\mu)^T\Sigma^{-1}(x-\mu)
=\\=&
\begin{bmatrix}
	x-\mu _1 & y-\mu _2 & z-\mu _3
\end{bmatrix}\cdot
\begin{bmatrix}
	\rho _{1,1} & \rho _{1,2} & \rho _{1,3} \\
	\rho _{1,2} & \rho _{2,2} & \rho _{2,3} \\
	\rho _{1,3} & \rho _{2,3} & \rho _{3,3}
\end{bmatrix}\cdot
\begin{bmatrix}
	x-\mu _1 \\
	y-\mu _2 \\
	z-\mu _3
\end{bmatrix}
=\\=&
\begin{bmatrix}
	x-\mu _1 & y-\mu _2 & z-\mu _3
\end{bmatrix}\cdot
\begin{bmatrix}
	\left(x-\mu _1\right) \rho _{1,1}+\left(y-\mu
   _2\right) \rho _{1,2}+\left(z-\mu _3\right) \rho
   _{1,3} \\
	\left(x-\mu _1\right) \rho _{1,2}+\left(y-\mu
   _2\right) \rho _{2,2}+\left(z-\mu _3\right) \rho
   _{2,3} \\
	\left(x-\mu _1\right) \rho _{1,3}+\left(y-\mu
   _2\right) \rho _{2,3}+\left(z-\mu _3\right) \rho
   _{3,3}
\end{bmatrix}
=\\=&
 \rho _{1,1} \left(\mu _1^2+x^2-2 \mu _1 x\right) \\
+&\rho _{1,2} \left(2 \mu _1 \mu _2-2 \mu _2 x+2 x y-2 \mu _1 y\right) \\
+&\rho _{1,3} \left(2 \mu _1 \mu _3-2 \mu _3 x+2 x z-2 \mu _1 z\right) \\
+&\rho _{2,2} \left(\mu _2^2+y^2-2 \mu _2 y\right) \\
+&\rho _{2,3} \left(2 \mu _2 \mu _3-2 \mu _3 y+2 y z-2 \mu _2 z\right) \\
+&\rho _{3,3} \left(\mu _3^2+z^2-2 \mu _3 z\right) \\
\end{align*}
Ponieważ nie ma wyrazu wolnego, oznacza to, że wektor $ \mu=\vec 0 $, co upraszcza obliczenia
\begin{align*}
&x^2 \rho _{1,1}+2 x y \rho _{1,2}+2 x z \rho _{1,3}+y^2 \rho _{2,2}+2 y z \rho _{2,3}+z^2 \rho _{3,3}
=\\=&
2 x^2-2 x y+4 x z+y^2-2 y z+3 z^2
\end{align*}
Spisując rozwiązania
\begin{gather*}
\Sigma^{-1}=
\begin{bmatrix}
	2  & -1 & 2  \\
	-1 & 1  & -1 \\
	2  & -1 & 3
\end{bmatrix}\\
\Sigma=
\begin{bmatrix}
	2  & 1 & -1 \\
	1  & 2 & 0  \\
	-1 & 0 & 1
\end{bmatrix}.
\end{gather*}

Oto szukana macierz kowariancji.\\
Aby znaleźć gęstość wektora $ Y $ musimy znaleźć parametry $ m_Y $ i $ \Sigma_Y $, gdzie
\begin{itemize}
\item $ m_Y=Am $,
\item $ \Sigma_Y=A\Sigma A^T $.
\end{itemize}
Ponieważ wektor $ m $ jest wektorem zerowym, to również wektor $ m_Y $ jest wektorem zerowym, zatem pozostaje skupić się na macierzy kowariancji.
\begin{gather*}
\Sigma_Y=A\Sigma A^T=
\begin{bmatrix}
	1 & 0  & 1 \\
	1 & -1 & 1 \\
	1 & 1  & 0
\end{bmatrix}\cdot
\begin{bmatrix}
	2  & 1 & -1 \\
	1  & 2 & 0  \\
	-1 & 0 & 1
\end{bmatrix}\cdot
\begin{bmatrix}
	1 & 1  & 1 \\
	0 & -1 & 1 \\
	1 & 1  & 0
\end{bmatrix}
=\\=
\begin{bmatrix}
	1 & 1  & 0  \\
	0 & -1 & 0  \\
	3 & 3  & -1
\end{bmatrix}\cdot
\begin{bmatrix}
	1 & 1  & 1 \\
	0 & -1 & 1 \\
	1 & 1  & 0
\end{bmatrix}=
\begin{bmatrix}
	1 & 0  & 2  \\
	0 & 1  & -1 \\
	2 & -1 & 6
\end{bmatrix}.
\end{gather*}
Do wyznaczenia gęstości potrzebna jest macierz $ \Sigma_Y^{-1} $
\begin{gather*}
\Sigma_Y^{-1}=
\begin{bmatrix}
	5  & -2 & -2 \\
	-2 & 2  & 1  \\
	-2 & 1  & 1
\end{bmatrix}.
\end{gather*}
Wykorzystując fakt, iż wektor średnich jest zerowy, możemy zapisać wzór na gęstość zmiennej losowej $ Y $ następująco
\begin{gather*}
f_Y(t)=\frac{1}{\sqrt{2\pi}^3}\det(\Sigma_Y)^{-\frac{1}{2}}
\exp\left(-\frac{1}{2}t^T\Sigma_Y^{-1}t\right),
\end{gather*}
gdzie $ t $ w naszym przypadku to $ t=(x,y,z) $.
Wykonajmy niezbędne obliczenia pomocnicze
\begin{align*}
&\det(\Sigma_Y^{-1})=1\\
&\begin{bmatrix}
	x & y & z
\end{bmatrix}\cdot
\begin{bmatrix}
	5  & -2 & -2 \\
	-2 & 2  & 1  \\
	-2 & 1  & 1
\end{bmatrix}\cdot
\begin{bmatrix}
	x \\
	y \\
	z
\end{bmatrix}
=\\=&
\begin{bmatrix}
	5 x-2 y-2 z & -2 x+2 y+z & -2 x+y+z
\end{bmatrix}\cdot
\begin{bmatrix}
	x \\
	y \\
	z
\end{bmatrix}
=\\=&
\begin{bmatrix}
	x (5 x-2 y-2 z)+z (-2 x+y+z)+y (-2 x+2 y+z)
\end{bmatrix}
=\\=&
\begin{bmatrix}
	5 x^2-4 x (y+z)+2 y^2+2 y z+z^2
\end{bmatrix}
\end{align*}
Ostatecznie
\begin{gather*}
f_Y(x,y,z)=\frac{1}{2\sqrt{2}\pi^{\frac{3}{2}}}\exp \left(5 x^2-4 x y-4 x z+2 y^2+2 y z+z^2\right)
\end{gather*}


\subsection*{Zadanie 62}
\addcontentsline{toc}{section}{Zadanie 62}
Dane są niezależne zmienne losowe $ X,Y,Z $ są o rozkładzie normalnym $ \mathcal N(0,1) $. Wykazać, że zmienna losowa
\begin{gather*}
U=\frac{X+YZ}{\sqrt{1+Z^2}}
\end{gather*}
ma rozkład normalny $ \mathcal N(0,1) $



\subsection*{Zadanie 65}
\addcontentsline{toc}{section}{Zadanie 65}
Zmienna losowa $ X $ ma rozkład $ t $ Studenta o $ n $ stopniach swobody, jeśli gęstość wynosi
\begin{gather*}
f(x)=\frac{1}{\sqrt nB \left(\frac{1}{2},\frac{n}{2}\right)}\left(1+\frac{x^2}{n}\right)^{-\frac{n+1}{2}},
\end{gather*}
gdzie $ B \left(\frac{1}{2},\frac{n}{2}\right) $ jest funkcją beta. Dane są niezależne zmienne losowe $ X $ i $ Y $ o rozkładach (odpowiednio) $ \mathcal N(0,1) $ i $ G\left(\frac{1}{2},\frac{n}{2}\right) $. Wykazać, że zmienna losowa
\begin{gather*}
Z=\frac{X}{\sqrt{\frac{Y}{n}}}
\end{gather*}
ma rozkład $ t$ Studenta o $ n $ stopnia swobody. Korzystając ze wzoru Stirlinga dla funkcji gamma
\begin{gather*}
\ln \Gamma(a)=\sqrt{2\pi}+\left(a-\frac{1}{2}\right)\ln a-a+\frac{\theta }{12a},\qquad
a>0,\quad 0<\theta<1.
\end{gather*}
Wykazać, że
\begin{gather*}
f(x)\xrightarrow[n\to \infty]{}\frac{1}{\sqrt{2\pi }}\exp \left(-\frac{x^2}{2}\right),\qquad x\in \mathbb R 
\end{gather*}

Rozwiązanie:
\begin{itemize}
\item $ f_X(x)=\frac{1}{\sqrt{2\pi }}\exp \left(-\frac{x^2}{2}\right) $
\item $ f_Y(y)=\left(\frac{1}{2}\right)^\frac{n}{2}\frac{y^{\frac{n}{2}-1}}{\Gamma\left(\frac{n}{2}\right)} \exp\left(-\frac{y}{2}\right)$
\end{itemize}
\begin{align*}
&F_Z(t)=P\left(Z\le t\right)
=
P\left(\frac{X}{\sqrt{\frac{Y}{n}}}\le t\right)
=
P\left(X\le t\sqrt{\frac{Y}{n}}\right)
=\\=&
\int\limits_{0}^{\infty }F_X\left(t\sqrt{\frac{y}{n}}\right)f_Y(y)\,dy
\end{align*}
Różniczkujemy obustronnie po $ t $
\begin{align*}
&\left(F_Z(t)\right)'=
\int\limits_{0}^{\infty }f_X\left(t\sqrt{\frac{y}{n}}\right)f_Y(y)\sqrt{\frac{y}{n}}\,dy\\
&f_Z(t)=\int\limits_{0}^{\infty }\frac{1}{\sqrt{2\pi}}\exp \left(-\frac{1}{2}t^2\cdot
\frac{y}{n}\right)\left(\frac{1}{2}\right)^\frac{n}{2}\frac{1}{\Gamma(\frac{n}{2})}y^{\frac{n}{2}-1}\exp \left(-\frac{y}{2}\right)\sqrt{\frac{y}{n}}\,dy
=\\=&
\frac{1}{\sqrt n}\frac{1}{\Gamma(\frac{n}{2})\Gamma(\frac{1}{2})}\int\limits_{0}^{\infty }\left(\frac{1}{2}\right)^\frac{n+1}{2}y^{\frac{n+1}{2}-1}\exp \left(-\frac{1}{2} y \left(\frac{t^2}{n}+1\right)\right)\,dy
=\\=&
\frac{1}{\sqrt n}\frac{\Gamma(\frac{n}{2}+\frac{1}{2})}{\Gamma(\frac{n}{2})\Gamma(\frac{1}{2})}\left(1+\frac{t^2}{n}\right)^{-\frac{n+1}{2}}
\cdot\\\cdot&
\int\limits_{0}^{\infty }\frac{1}{\Gamma(\frac{n}{2}+\frac{1}{2})}\left(\frac{1}{2}\right)^\frac{n+1}{2}\left(1+\frac{t^2}{n}\right)^{\frac{n+1}{2}}y^{\frac{n+1}{2}-1}\exp \left(-\frac{1}{2} y \left(\frac{t^2}{n}+1\right)\right)\,dy
=\\=&
\frac{1}{\sqrt n}\frac{1}{B(\frac{1}{2},\frac{n}{2})}\left(1+\frac{t^2}{n}\right)^{-\frac{n+1}{2}}
\end{align*}
Do dokończenia


\subsection*{Zadanie 66}
\addcontentsline{toc}{section}{Zadanie 66}
Zmienna losowa $ X $ ma rozkład beta $ B(p,q) $, jeśli jej gęstość wynosi
\begin{gather*}
f(x)\frac{1}{B(p,q)}x^{p-1}(1-x)^{q-1}
\end{gather*}
gdzie $ p>0,q>0 $ oraz $ B(p,q) $ jest funkcją beta. Udowodnić że jeśli zmienna losowa $ X $ ma rozkład $ t $ Studenta o $ n $ stopniach swobody to zmienna losowa
\begin{gather*}
Y=\frac{1}{1+\frac{X^2}{n}}
\end{gather*}
ma rozkład beta $ B\left(\frac{n}{2},\frac{1}{2}\right) $.

Rozwiązanie:
\begin{itemize}
\item Gęstość rozkładu $ t $ Studenta o $ n $ stopniach swobody
\begin{gather*}
f(x)=\frac{1}{\sqrt nB \left(\frac{1}{2},\frac{n}{2}\right)}\left(1+\frac{x^2}{n}\right)^{-\frac{n+1}{2}}
\end{gather*}
\end{itemize}
Jeżeli
\begin{gather*}
Y=\frac{1}{1+\frac{X^2}{n}}
\end{gather*}
to
\begin{gather*}
X=\pm\sqrt n\sqrt{\frac{1}{Y}-1}
\end{gather*}
Policzmy jeszcze potrzebną później pochodną
\begin{gather*}
\left(\sqrt n\sqrt{\frac{1}{y}-1}\right)'
=\\=
-\frac{\sqrt{n}}{2 \sqrt{\frac{1}{y}-1} y^2}
=\\=
-\frac{\sqrt{n}}{2\sqrt{1-y}} \left(\frac{1}{y}\right)^{\frac{3}{2}}
=\\=
-\frac{\sqrt{n}}{2\sqrt{1-y}} y^{-\frac{3}{2}}
\end{gather*}
Następnie
\begin{align*}
&f_Y(y)
=\\=&
2f_X\left(\sqrt n\sqrt{\frac{1}{y}-1}\right)\left|-\frac{\sqrt{n}}{2\sqrt{1-y}} y^{-\frac{3}{2}}\right|
=\\=&
\frac{1}{\sqrt nB \left(\frac{1}{2},\frac{n}{2}\right)}y^{\frac{n+1}{2}}\frac{\sqrt{n}}{\sqrt{1-y}} y^{-\frac{3}{2}}
=\\=&
\frac{1}{B \left(\frac{1}{2},\frac{n}{2}\right)}y^{\frac{n+1}{2}}\frac{1}{\sqrt{1-y}} y^{-\frac{3}{2}}
=\\=&
\frac{1}{B \left(\frac{1}{2},\frac{n}{2}\right)}y^{\frac{n}{2}-1}\left(1-y\right)^{\frac{1}{2}-1}
\end{align*}
I gotowe.


\subsection*{Zadanie 67}
\addcontentsline{toc}{section}{Zadanie 67}
Wykazać, że jeśli zmienne losowe $ X $ i $ Y $ są niezależne o rozkładach gamma odpowiednio $ G(1,p_1) $ i $ G(1,p_2) $ to zmienna losowa
\begin{gather*}
Z=\frac{X}{X+Y}
\end{gather*}
ma rozkład beta $ B (p_1,p_2) $.

Rozwiązanie:
\begin{itemize}
\item 
\begin{gather*}
f_X(x)=\frac{1}{\Gamma(p_1)}x^{p_1-1}\exp \left(-x\right)
\end{gather*}
\item 
\begin{gather*}
f_Y(y)=\frac{1}{\Gamma(p_2)}y^{p_2-1}\exp \left(-y\right)
\end{gather*}
\end{itemize}
Przekształćmy dystrybuantę zmiennej $ Z $
\begin{align*}
&F_Z(t)=
P\left(Z\le t\right)=
P\left(\frac{X}{X+Y}\le t\right)=
P\left(X\le t(X+Y)\right)
=\\=&
P\left(X(1-t)\le tY\right)=
P\left(X\le \frac{t}{1-t}Y\right)=F_X\left(\frac{t}{1-t}Y\right)
=\\=&
\int\limits_{0}^{\infty }F_X\left(\frac{t}{1-t}y\right)f_Y(y)\,dy
\end{align*}
Różniczkujemy po $ t $
\begin{align*}
&f_Z(t)=\left(F_Z(t)\right)'=F_X'\left(\frac{t}{1-t}y\right)
=\\=&
\left(\int\limits_{0}^{\infty }F_X\left(\frac{t}{1-t}y\right)f_Y(y)\,dy\right)'
=\\=&
\int\limits_{0}^{\infty }f_X\left(\frac{t}{1-t}y\right)f_Y(y)\frac{Y}{(t-1)^2}\,dy
=\\=&
\int\limits_{0}^{\infty }
\frac{1}{\Gamma(p_1)}
\left(\frac{t}{1-t}y\right)^{p_1-1}
e^{-\frac{t}{1-t}y}
\frac{1}{\Gamma(p_2)}
y^{p_2-1}
e^{-y}
\frac{y}{(t-1)^2}
\,dy
=\\=&
\frac{1}{\Gamma(p_1)\Gamma(p_2)}
t^{p_1-1}
\int\limits_{0}^{\infty }
\left(\frac{1}{1-t}\right)^{p_1+1}
y^{p_1+p_2-1}
e^{-\frac{1}{1-t}y}
\,dy
=\\=&
\frac{\Gamma(p_1+p_2)}{\Gamma(p_1)\Gamma(p_2)}
t^{p_1-1}
\left(1-t\right)^{p_2-1}
\int\limits_{0}^{\infty }
\frac{y^{p_1+p_2-1}}{\Gamma(p_1+p_2)}
\left(\frac{1}{1-t}\right)^{p_1+p_2}
e^{-\frac{1}{1-t}y}
\,dy
=\\=&
\frac{\Gamma(p_1+p_2)}{\Gamma(p_1)\Gamma(p_2)}
t^{p_1-1}
\left(1-t\right)^{p_2-1}
\end{align*}\qed



\subsection*{Zadanie 68}
\addcontentsline{toc}{section}{Zadanie 68}
Zmienna losowa $ X $ ma rozkład $ F $ Snedecora o $ (n,m) $ stopniach swobody, jeśli jej gęstość wynosi
\begin{gather*}
f(x)=\frac{1}{B \left(\frac{n}{2},\frac{m}{2}\right)}\left(\frac{n}{m}\right)^\frac{n}{2}x^{\frac{n}{2}-1}\left(1+\frac{n}{m}x\right)^{-\frac{n+m}{2}}\mathbbm1_{[0,+\infty )}(x)
\end{gather*}
gdzie $ B\left(\frac{n}{2},\frac{m}{2}\right) $ jest funkcją beta. Wykazać, że jeśli zmienna losowa $ X $ ma rozkład $ \chi^2 $ o $ n $ stopniach swobody i zmienna losowa $ Y $ ma rozkład $ \chi^2 $ o $ m $ stopniach swobody i zmienne te są niezależne to zmienna losowa
\begin{gather*}
F=\frac{Xm}{Yn}
\end{gather*}
ma rozkład $ F $ Snedecora o $ (n,m) $ stopniach swobody.

Rozwiązanie:\\
Gęstości zmiennych losowych $ X $ i $ Y $ mają postać
\begin{align*}
f_X(x)=\frac{1}{\Gamma(\frac{n}{2})}\left(\frac{x}{2}\right)^{\frac{n}{2}-1}\exp \left(-\frac{x}{2}\right)
&&
f_Y(y)=\frac{1}{\Gamma(\frac{n}{2})}\left(\frac{y}{2}\right)^{\frac{n}{2}-1}\exp \left(-\frac{y}{2}\right)
\end{align*}
Wyznaczmy rozkład $ F $
\begin{gather*}
F_F(t)=
P\left(F\le t\right)=
P\left(\frac{Xm}{Yn}\le t\right)=
P\left(X\le t\frac{Yn}{m}\right)=F_X\left(t\frac{Yn}{m}\right)
=\\=
\int\limits_{0}^{\infty }
F_X\left(\frac{ytn}{m}\right)
\frac{1}{\Gamma(\frac{n}{2})}\left(\frac{y}{2}\right)^{\frac{n}{2}-1}\exp \left(-\frac{y}{2}\right)
\end{gather*}
Różniczkujemy obustronnie po $ t $
\begin{align*}
&\left(F_F(t)\right)'=f_F(t)
=\\=&
\int\limits_{0}^{\infty }
F_X'\left(\frac{ytn}{m}\right)
\frac{1}{\Gamma(\frac{n}{2})}y^{\frac{m}{2}-1}\left(\frac{1}{2}\right)^{\frac{m}{2}-1}\exp \left(-\frac{y}{2}\right)\,dy
=\\=&
\int\limits_{0}^{\infty }
\frac{1}{\Gamma(\frac{m}{2})}\left(\frac{ytn}{2m}\right)^{\frac{n}{2}-1}\exp \left(-\frac{ytn}{2m}\right)
\frac{1}{\Gamma(\frac{n}{2})}y^{\frac{m}{2}-1}\left(\frac{1}{2}\right)^{\frac{m}{2}-1}\exp \left(-\frac{y}{2}\right)\frac{yn}{m}\,dy
=\\=&
\frac{1}{\Gamma(\frac{m}{2})\Gamma(\frac{n}{2})}
\int\limits_{0}^{\infty }
\left(\frac{ytn}{2m}\right)^{\frac{n}{2}-1}
y^{\frac{m}{2}-1}\left(\frac{1}{2}\right)^{\frac{m}{2}-1}\exp \left(-\frac{y}{2}\left(1+\frac{n t}{m}\right)\right)\frac{yn}{m}\,dy
=\\=&
%\frac{\Gamma\left(\frac{m}{2}+\frac{n}{2}\right)}{\Gamma(\frac{m}{2})\Gamma(\frac{n}{2})}t^{\frac{n}{2}-1}\left(\frac{n}{m}\right)^{\frac{n}{2}}
%\int\limits_{0}^{\infty }
%\frac{1}{\Gamma\left(\frac{m}{2}+\frac{n}{2}\right)}
%\left(\frac{y}{2}\right)^{\frac{n}{2}-1}
%y^{\frac{m}{2}-1}\left(\frac{1}{2}\right)^{\frac{m}{2}-1}\exp \left(-\frac{y}{2}\left(1+\frac{n t}{m}\right)\right)y\,dy
%=\\=&
%\frac{\Gamma\left(\frac{m}{2}+\frac{n}{2}\right)}{\Gamma(\frac{m}{2})\Gamma(\frac{n}{2})}t^{\frac{n}{2}-1}\left(\frac{n}{m}\right)^{\frac{n}{2}}
%\int\limits_{0}^{\infty }
%\frac{1}{\Gamma\left(\frac{m}{2}+\frac{n}{2}\right)}
%y^{\frac{n+m}{2}-1}\left(\frac{1}{2}\right)^{\frac{n+m}{2}-2}\exp \left(-\frac{y}{2}\left(1+\frac{n t}{m}\right)\right)\,dy
\end{align*}


\subsection*{Zadanie 69}
\addcontentsline{toc}{section}{Zadanie 69}
Wykazać, że jeżeli zmienna losowa $ X $ ma rozkład $ t $ Studenta o $ n $ stopniach swobody, to zmienna losowa $ Y=X^2 $  ma rozkład $ F $ Snedecora o $ (1,n) $ stopniach swobody.

Rozwiązanie:
\begin{itemize}
\item Gęstość rozkładu $ t $ Studenta o $ n $ stopniach swobody
\begin{gather*}
f_X(x)=
\frac{\Gamma\left(\frac{n+1}{2}\right)}{\Gamma\left(\frac{1}{2}\right)\Gamma\left(\frac{n}{2}\right)\sqrt n}\left(1+\frac{x^2}{n}\right)^{-\frac{n+1}{2}}
\end{gather*}
\item Gęstość zmiennej losowej o rozkładzie $ F $ Snedecora o $ (n,m) $ stopniach swobody
\begin{gather*}
f(y)=\frac{1}{B \left(\frac{n}{2},\frac{m}{2}\right)}\left(\frac{n}{m}\right)^\frac{n}{2}y^{\frac{n}{2}-1}\left(1+\frac{n}{m}y\right)^{-\frac{n+m}{2}}
\end{gather*}
\item Gęstość zmiennej losowej o rozkładzie $ F $ Snedecora o $ (1,n) $ stopniach swobody
\begin{gather*}
f(y)=
\frac{n^{\frac{n}{2}} (n+x)^{-\frac{n+1}{2}}}
{\sqrt{x}B\left(\frac{1}{2},\frac{n}{2}\right)}
\end{gather*}
\end{itemize}
Przekształćmy dystrybuantę
\begin{align*}
&F_Y(t)=
P\left(Y\le t\right)=
P\left(X^2\le t\right)=
P\left(X\le \sqrt t\right)=F_X\left(\sqrt t\right)
\end{align*}
Różniczkując po $ t $ otrzymamy gęstość
\begin{align*}
&f_Y(t)=\left(F_X\left(\sqrt t\right)\right)'=
f_X\left(\sqrt t\right)
\frac{1}{2\sqrt t}
=\\=&
\frac{\Gamma\left(\frac{n+1}{2}\right)}{\Gamma\left(\frac{1}{2}\right)\Gamma\left(\frac{n}{2}\right)\sqrt n}\left(1+\frac{t}{n}\right)^{-\frac{n+1}{2}}
\frac{1}{2\sqrt t}
=\\=&
\frac{1}{B\left(\frac{1}{2},\frac{n}{2}\right)\sqrt n}
\left(\frac{n}{n}\right)^{-\frac{n+1}{2}}
\left(1+\frac{t}{n}\right)^{-\frac{n+1}{2}}
\frac{1}{2\sqrt t}
=\\=&
\frac{n^\frac{n}{2}}{B\left(\frac{1}{2},\frac{n}{2}\right)\sqrt t}
\left(n+t\right)^{-\frac{n+1}{2}}
\end{align*}\qed

\subsection*{Zadanie 70}
\addcontentsline{toc}{section}{Zadanie 70}
Wykazać, że jeśli zmienna losowa $ X $ ma rozkład $ F $ Snedecora o $ (m,n) $ stopniach swobody to zmienna
\begin{gather*}
Y=\frac{mX}{n+mX}
\end{gather*}
am rozkład beta $ B \left(\frac{m}{2},\frac{n}{2}\right) $

Rozwiązanie:
\begin{itemize}
\item Gęstość rozkładu beta
\begin{gather*}
f(x)=\frac{1}{B (p,q)}x^{p-1}(1-x)^{q-1}
\end{gather*}
\item Gęstość rozkładu $ F $ Snedecora
\begin{gather*}
f(x)=\frac{1}{B \left(\frac{n}{2},\frac{m}{2}\right)}\left(\frac{n}{m}\right)^\frac{n}{2}x^{\frac{n}{2}-1}\left(1+\frac{n}{m}x\right)^{-\frac{n+m}{2}}\mathbbm1_{[0,+\infty )}(x)
\end{gather*}
\end{itemize}
Przekształćmy
\begin{align*}
&F_Y(t)=
P\left(Y\le t\right)=
P\left(\frac{mX}{n+mX}\le t\right)=
P\left(mX\le t(n+mX)\right)
=\\=&
P\left(mX-tmX\le tn\right)=
P\left(X(m-tm)\le tn\right)=
P\left(X\le \frac{tn}{m-tm}\right)
=\\=&
F_X(\frac{tn}{m-tm})
\end{align*}
Pochodna po $ t $
\begin{align*}
&F_X'\left(\frac{tn}{m-tm}\right)=
f_X\left(\frac{tn}{m-tm}\right)\frac{n}{m (t-1)^2}=f_Y(t)
\end{align*}
Przekształcamy i podstawiamy
\begin{align*}
&f_X\left(\frac{tn}{m-tm}\right)\frac{n}{m (t-1)^2}
=\\=&
\frac{1}{B \left(\frac{n}{2},\frac{m}{2}\right)}
\left(\frac{m}{n}\right)^\frac{m}{2}
\left(\frac{tn}{m-tm}\right)^{\frac{m}{2}-1}
\left(1+\frac{m}{n}\cdot \frac{tn}{m-tm}\right)^{-\frac{n+m}{2}}
\frac{n}{m (t-1)^2}
=\\=&
\frac{1}{B \left(\frac{n}{2},\frac{m}{2}\right)}
\left(\frac{m}{n}\right)^\frac{m}{2}
\left(\frac{tn}{m(1-t)}\right)^{\frac{m}{2}-1}
\left(\frac{1}{1-t}\right)^{-\frac{n+m}{2}}
\frac{n}{m (t-1)^2}
=\\=&
\frac{1}{B \left(\frac{n}{2},\frac{m}{2}\right)}
\left(\frac{m}{n}\right)^\frac{m}{2}
\left(\frac{n}{m}\right)^{\frac{m}{2}-1}
\left(\frac{t}{1-t}\right)^{\frac{m}{2}-1}
\left(\frac{1}{1-t}\right)^{-\frac{n+m}{2}}
\frac{n}{m (t-1)^2}
=\\=&
\frac{1}{B \left(\frac{n}{2},\frac{m}{2}\right)}
\left(\frac{t}{1-t}\right)^{\frac{m}{2}-1}
\left(\frac{1}{1-t}\right)^{-\frac{n+m}{2}}
\frac{1}{(t-1)^2}
=\\=&
\frac{1}{B \left(\frac{n}{2},\frac{m}{2}\right)}
t^{\frac{m}{2}-1}
\left(1-t\right)^{\frac{n}{2}-1}
\end{align*}

Inaczej. Jeśli
\begin{gather*}
Y=\frac{mX}{n+mX}
\end{gather*}
otrzymujemy
\begin{gather*}
X=\frac{n Y}{m (1-Y)}
\end{gather*}
Potrzebna pochodna
\begin{gather*}
\left(\frac{n Y}{m (1-Y)}\right)'=
\frac{n}{m (1-Y)^2}
\end{gather*}
Policzmy gęstość
\begin{align*}
&f_Y(y)
=\\=&
f_X\left(\frac{n y}{m (1-y)}\right)\left|\frac{n}{m (1-Y)^2}\right|
=\\=&
\frac{1}{B \left(\frac{m}{2},\frac{n}{2}\right)}\left(\frac{m}{n}\right)^\frac{m}{2}\left(\frac{n y}{m (1-y)}\right)^{\frac{m}{2}-1}\left(1-y\right)^{\frac{n+m}{2}}\frac{n}{m (1-y)^2}
=\\=&
\frac{1}{B \left(\frac{m}{2},\frac{n}{2}\right)}\left(\frac{y}{1-y}\right)^{\frac{m}{2}-1}\left(1-y\right)^{\frac{n+m}{2}} (1-y)^{-2}
=\\=&
\frac{1}{B \left(\frac{m}{2},\frac{n}{2}\right)}y^{\frac{m}{2}-1}\left(1-y\right)^{\frac{n+m}{2}-\frac{m}{2}+1-2}
=\\=&
\frac{1}{B \left(\frac{m}{2},\frac{n}{2}\right)}y^{\frac{m}{2}-1}\left(1-y\right)^{\frac{n}{2}-1}
\end{align*}
Krócej.


\subsection*{Zadanie 71}
\addcontentsline{toc}{section}{Zadanie 71}
Niech $ X=(X_1,\dots,X_n) $ będzie próbą z rozkładu normalnego $ \mathcal N(m,\sigma^2) $. Wykazać, że statystyki
\begin{align*}
\overline X=\frac{1}{n}\sum_{i=1}^{n}X_i
&&
S^2=\frac{1}{n}\sum_{i=1}^{n}\left(X_i-\overline X\right)^2
\end{align*}
są niezależne (jako zmienne losowe) oraz $ \overline X $ ma rozkład normalny $ \mathcal N(m,\frac{\sigma ^2}{n}) $, a Statystyka $ \frac{nS^2}{\sigma^2} $ ma rozkład $ \chi^2 $ o $ n-1 $ stopniach swobody.

Rozwiązanie:
\begin{itemize}
\item $ \varphi_{X_i}=\exp\left(itm-\frac{t^2\sigma^2}{2}\right) $
\end{itemize}
Tymczasowo pomińmy badanie niezależności i przejdźmy do wyznaczania rozkładów poprzez wykorzystanie funkcji charakterystycznej.
\begin{gather*}
\mathbb E e^{it\overline{X}}=
\mathbb E \exp\left(\frac{it}{n}\sum_{i=1}^{n}X_i\right)=
\mathbb E \prod_{i=1}^{n}\exp\left(\frac{it}{n}X_i\right)=
\prod_{i=1}^{n}\mathbb E \exp\left(\frac{it}{n}X_i\right)
=\\=
\left(\mathbb E \exp\left(\frac{it}{n}X_i\right)\right)^n=
\exp\left(i m t+\frac{\sigma ^2 t^2}{2 n}\right)
\end{gather*}
Jest to funkcja charakterystyczna rozkładu normalnego $ \mathcal N(m,\frac{\sigma^2}{n}) $. Zajmijmy się w podobny sposób drugą statystyką.
Na początek zbadajmy rozkład zmiennej losowej $ \overline X - X_i$


\begin{align*}
&\varphi_{\overline X-X_i}(t)
=\\=&
\varphi_{\frac{1}{n}\sum_{k=1}^{n}X_k-X_i}(t)
=\\=&
\varphi_{\frac{1}{n}\sum_{\substack{k=1\\k\neq i}}^{n}X_k-\frac{n-1}{n}X_i}(t)
=\\=&
\varphi_{\frac{1}{n}\sum_{\substack{k=1\\k\neq i}}^{n}X_k}(t)\varphi_{-\frac{n-1}{n}X_i}(t)
=\\=&
\varphi_{\sum_{\substack{k=1\\k\neq i}}^{n}X_k}\left(\tfrac{t}{n}\right)
\varphi_{X_i}\left(-\frac{n-1}{n}t\right)
=\\=&
\exp \left(itm\frac{n-1}{n} -\frac{\sigma ^{2}t^{2}}{2}\cdot \frac{n-1}{n^2}\right)
\exp\left(-itm\frac{n-1}{n}-\frac{\sigma^2t^2}{2}\cdot \left(\frac{n-1}{n}\right)^2\right)
=\\=&
\exp \left(-\frac{\sigma ^2t^2}{2}\cdot\frac{n-1}{n}\right)
\end{align*}
Wyrażenie w nawiasie ma rozkład normalny $ \mathcal N(0,\frac{n-1}{n}\sigma ^2) $. Oznaczmy\\ $ Y_i=\sigma\sqrt{\frac{n}{n-1}}\left(X_i-\overline X\right) $, które mają rozkład standardowy normalny. Wtedy
\begin{align*}
&\frac{nS^2}{\sigma^2}=
\frac{1}{\sigma ^2}\sum_{i=1}^{n}\left(\sigma\sqrt{\frac{n-1}{n}}Y_i\right)^2
=\\=&
\sum_{i=1}^{n}\frac{n-1}{n}Y_i^2
\end{align*}
Przejdźmy na funkcję charakterystyczną
\begin{align*}
&\varphi_{\frac{nS^2}{\sigma^2}}(t)
=\\=&
\end{align*}


%\begin{align*}
%&P\left(\overline X-X_i\le t\right )
%=\\=&
%P\left(\frac{1}{n}\sum_{k=1}^{n}X_k-X_i\le t\right)
%=\\=&
%\int\limits_{\mathbb R }P\left(\frac{1}{n}\sum_{k=1}^{n}X_k-X_i\le t|X_i=x\right)f(x)\,dx
%=\\=&
%\int\limits_{\mathbb R }
%P\left(\frac{1}{n}\sum_{k=1}^{n}X_k\le t+x|X_i=x\right)
%f(x)\,dx
%=\\=&
%\int\limits_{\mathbb R }
%P\left(\frac{1}{n}\sum_{\substack{k=1\\k\neq i}}^{n}X_k\le t+x\frac{n-1}{n}|X_i=x\right)
%f(x)\,dx
%=\\=&
%\int\limits_{\mathbb R }
%P\left(\frac{1}{n-1}\sum_{\substack{k=1\\k\neq i}}^{n}X_k\le t\frac{n}{n-1}+x|X_i=x\right)
%f(x)\,dx
%=\\=&
%\int\limits_{\mathbb R }
%P\left(\frac{1}{n-1}\sum_{\substack{k=1\\k\neq i}}^{n}X_k\le t\frac{n}{n-1}+x\right)
%f(x)
%\,dx
%\end{align*}

\begin{align*}
\mathbb E e^{it\frac{nS^2}{\sigma^2}}=&
\mathbb E \exp \left(it\frac{n}{\sigma ^2}\cdot\frac{1}{n}\sum_{i=1}^{n}\left(X_i-\overline X\right)^2\right)
=\\=&
\mathbb E \exp \left(it\frac{1}{\sigma ^2}\sum_{i=1}^{n}\left(X_i-\overline X\right)^2\right)
=\\=&
\mathbb E \exp \left(\frac{it}{\sigma ^2}\left(X_i-\overline X\right)^2\right)
=\\=&
\mathbb E \exp \left(\frac{it}{\sigma ^2}\sum_{i=1}^{n}\left(\overline{X}^2-2 \overline{X} X_i+X_i^2\right)\right)
=\\=&
\mathbb E \exp \left(\frac{it}{\sigma ^2}\left(\sum_{i=1}^{n}\overline{X}^2-\sum_{i=1}^{n}2 \overline{X} X_i+\sum_{i=1}^{n}X_i^2\right)\right)
=\\=&
\mathbb E \exp \left(
\frac{it}{\sigma ^2}\left(
\frac{1}{n}\sum_{i=1}^{n}\sum_{k=1}^{n}X_iX_k
-\frac{2}{n}\sum_{i=1}^{n}\sum_{k=1}^{n}X_iX_k
+\sum_{i=1}^{n}X_i^2
\right)\right)
=\\=&
\mathbb E \exp \left(
\frac{it}{\sigma ^2}\left(
\sum_{i=1}^{n}X_i^2
-\frac{1}{n}\sum_{i=1}^{n}\sum_{k=1}^{n}X_iX_k
\right)\right)
\end{align*}



\subsection*{Zadanie 72}
\addcontentsline{toc}{section}{Zadanie 72}
Niech $ X=(X_1,\dots,X_n) $ będzie próbą z rozkładu normalnego $ \mathcal N(m,\sigma^2) $. Wykazać, że statystyka $ t $ Studenta określona wzorem
\begin{gather*}
t_n=\frac{\overline X-m}{S}\sqrt{n-1}
\end{gather*}
ma rozkład $ t $ Studenta o $ n-1 $ stopniach swobody.

Rozwiązanie:
\begin{itemize}
\item Gęstość rozkładu $ t $ Studenta o 
\begin{gather*}
f(x)=\frac{\Gamma(\frac{n+1}{2})} {\sqrt{n\pi}\,\Gamma(\frac{n}{2})} \left(1+\frac{x^2}{n} \right)^{-\left(\frac{n+1}{2}\right)}
\end{gather*}
\end{itemize}
Pamiętajmy, że $ \overline X $ oraz $ S $ są niezależne.
Policzmy
\begin{gather*}
\frac{\overline X-m}{S}\sqrt{n-1}=
\frac{\overline X-m}{\sigma}\sqrt n
\frac{1}{\sqrt{\frac{nS^2}{\sigma^2}}}\sqrt{n-1}
\end{gather*}
Zauważmy, że $ \frac{\overline X-m}{\sigma}\sqrt n $ ma rozkład $ \mathcal N(0,1) $, natomiast $ \frac{nS^2}{\sigma^2} $ ma rozkład $ \chi^2 $ o $ n-1 $. Oznaczmy je odpowiednio $ A $ i $ B $
\begin{gather*}
P\left(t_n\le t\right)=
P\left(\frac{A}{\sqrt B}\sqrt{n-1}\le t\right)=
P\left(A\le t\frac{\sqrt B}{\sqrt{n-1}}\right)
=\\=
\int\limits_{0}^{\infty }F\left(t\frac{\sqrt b}{\sqrt{n-1}}\right)
\left(\frac{1}{2}\right)^{\frac{n-1}{2}}\frac{b^{\frac{n-1}{2}-1}}{\Gamma\left(\frac{n-1}{2}\right)} \exp\left(-\frac{b}{2}\right)\,db
\end{gather*}
\begin{align*}
&f(t)=
\int\limits_{0}^{\infty }
F'\left(t\frac{\sqrt b}{\sqrt{n-1}}\right)
\left(\frac{1}{2}\right)^{\frac{n-1}{2}}
\frac{b^{\frac{n-1}{2}-1}}{\Gamma\left(\frac{n-1}{2}\right)}
\exp\left(-\frac{b}{2}\right)\,db
=\\=&
\int\limits_{0}^{\infty }
f\left(t\frac{\sqrt b}{\sqrt{n-1}}\right)
\left(\frac{1}{2}\right)^{\frac{n-1}{2}}
\frac{b^{\frac{n-1}{2}-1}}{\Gamma\left(\frac{n-1}{2}\right)}
\exp\left(-\frac{b}{2}\right)
\frac{\sqrt b}{\sqrt{n-1}}\,db
=\\=&
\int\limits_{0}^{\infty }
\frac{1}{\sqrt{2\pi}}\exp\left(-\frac{t^2}{2}\cdot \frac{b}{n-1}\right){\sqrt{n-1}}
\left(\frac{1}{2}\right)^{\frac{n-1}{2}}
\frac{b^{\frac{n-1}{2}-1}}{\Gamma\left(\frac{n-1}{2}\right)}
\exp\left(-\frac{b}{2}\right)
\frac{\sqrt b}{\sqrt{n-1}}\,db
=\\=&
\int\limits_{0}^{\infty }
\frac{1}{\sqrt{2\pi}}
\left(\frac{1}{2}\right)^{\frac{n-1}{2}}
\frac{b^{\frac{n-1}{2}-1}}{\Gamma\left(\frac{n-1}{2}\right)}
\exp\left(-\frac{b}{2}\left(1+\frac{t^2}{n-1}\right)\right)
\frac{\sqrt b}{\sqrt{n-1}}\,db
=\\=&
\frac{1}{\sqrt{\pi}}
\left(1+\frac{t^2}{n-1}\right)^{-\frac{n}{2}}\frac{\Gamma(\frac{n}{2})}{\Gamma\left(\frac{n-1}{2}\right)}
\int\limits_{0}^{\infty }
\left(1+\frac{t^2}{n-1}\right)^{\frac{n}{2}}
\left(\frac{1}{2}\right)^{\frac{n}{2}}
\frac{b^{\frac{n}{2}-1}}{\Gamma\left(\frac{n-1}{2}\right)}
\exp\left(-\frac{b}{2}\left(1+\frac{t^2}{n-1}\right)\right)
\frac{1}{\sqrt{n-1}}\,db
=\\=&
\frac{\Gamma\left(\frac{n}{2}\right)}{\Gamma\left(\frac{1}{2}\right)\Gamma\left(\frac{n-1}{2}\right)}
\left(1+\frac{t^2}{n-1}\right)^{-\frac{n}{2}}
\end{align*}