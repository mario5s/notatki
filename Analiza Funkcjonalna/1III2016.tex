\chapter{1 marca 2016}
$ \mathbb K $ - ciało liczbowe, $ \mathbb R $, $ \mathbb C $
\begin{defi}
Przestrzenią wektorową (liniową) nad ustalonym ciałem $ \mathbb bK $ nazywamy strukturę algebraiczną\begin{gather*}
\bigl(\mathfrak X,+,\left(\mathbb K,+,\cdot,0,1\right),\cdot\bigr),
\end{gather*}
gdzie:
\begin{itemize}
\item $ \mathfrak X $ - przestrzeń wektorowa
\item + - dodawanie wektorów
\item * - opis ciała $ \mathbb K $
\item $ \cdot $ - mnożenie wektorów przez skalar
\end{itemize}
spełniają:
\begin{enumerate}
\item $ (\mathfrak X,+) $ - grupa przemienna ze względu na dodawania wektorów
\begin{gather*}
x,y\in\mathfrak X\to x+y\in \mathfrak X\text{ jest wykonywalne}
\end{gather*}
\begin{align*}
&\forall_{x,y,z\in\mathfrak X}\,(x+y)+z=x+(y+z)\\

\end{align*}
\end{enumerate}
\end{defi}