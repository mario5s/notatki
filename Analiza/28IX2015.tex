\chapter{28 września 2015}
\begin{defi}
Mówimy, że ciąg $ \left(f_n\right) $ jest zbieżny punktowo do funkcji $ g:P\rightarrow \mathbb R $ wtedy i tylko wtedy, gdy:
\begin{gather*}
\forall_{x\in P}\forall_{\varepsilon>0}\exists_{n_o\in \mathbb{N}}\forall_{n>n_o}
\left|f_n(x)-g(x)\right|<\varepsilon
\end{gather*}
\end{defi}
\begin{prz}\textbf{ }
\begin{enumerate}[a)]
\item 
\begin{gather*}
P=\mathbb R \\
f_n(x)=\frac{x}{n}\\
f_n(x_0)=\frac{x_0}{n}\to0\Rightarrow g(x)=0
\end{gather*}
\item 
\begin{gather*}
P=\mathbb R \\
f_n(x)=xe^{-nx}\\
\forall_x \lim\limits_{n\to\infty } f_n(x)=0
\end{gather*}
\item 
\begin{gather*}
P=[0,1]\\
f_n(x)=x^n\\
g(x)=\left \{
\begin{array}{ll}
0, & x\in[0,1)\\
1, &x=1
\end{array}
\right .
\end{gather*}
\end{enumerate}
\end{prz}
\begin{defi}
Ciąg $ \left(f_n\right) $ jest zbieżny jednostajnie na zbiorze $ P $ do funkcji $ g:P\rightarrow \mathbb R  $ wtedy i tylko wtedy, gdy:
\begin{gather*}
\forall_{\varepsilon>0}
\exists_{n_o\in \mathbb{N}}
\forall_{n>n_o}
\forall_{x\in P}
\left|f_n(x)-g(x)\right|<\varepsilon
\end{gather*}
\end{defi}
\begin{twr}
Założenia:
\begin{itemize}
\item $ \left(f_n\right) $ funkcje ciągłe
\item $ f_n\to f $ jednostajnie zbieżny na $ P $
\end{itemize}
Teza:\\
\begin{gather*}
f:P\rightarrow \mathbb R  \text{ jest ciągła}
\end{gather*}
\end{twr}
\begin{gather*}
\mathcal C \left(\left[a,b\right],\mathbb R \right)=
\left\{f:\left[a,b\right]\to \mathbb R ;\text{ ciągła}\right\}
\end{gather*}
\begin{itemize}
\item przestrzeń liniowa ($ f+g;\alpha\cdot f $)
\item przestrzeń unormowana $ \left\|f\right\|_0=\sup_{x\in\left[a,b\right]}\left|f(x)\right| $
\end{itemize}
\begin{twr}
Ciąg $ \left(f_n\right)\in \mathcal C \left(\left[a,b\right],\mathbb R \right) $ jest zbieżny do $ f\in \mathcal C  $ wtedy i tylko wtedy, gdy:
\begin{gather*}
\lim\limits_{n\to\infty }\left\|f_n-f\right\|_0=0
\end{gather*}
\end{twr}
\begin{proof}
Mamy pokazać, że $ f $ jest ciągła w $ x_0 $.
\begin{gather*}
\forall_{\varepsilon>0}
\exists_{\delta>0}
\forall_{x\in P}
\left|x-x_0\right|<\delta
\Rightarrow
\left|f(x)-f(x_0)\right|<\varepsilon
\end{gather*}
$ \varepsilon>0 $ ustalone:
\begin{align*}
&\exists_\delta
\forall_x
\left|x-x_0\right|<\delta\Rightarrow \left|f_n(x)-f_n(x_0)\right|<\tfrac{\varepsilon}{3}\\
&\exists_{n_o}
\forall_{n>n_0}
\forall_x
\left|f_n(x)-f(x)\right|<\tfrac{\varepsilon}{3}
\end{align*}
\begin{gather*}
\left|f(x)-f(x_0)\right|
\le
\left|f(x)-f_n(x)\right|
+
\left|f_n(x)-f_n(x_0)\right|
+
\left|f_n(x_0)-f(x_0)\right|<\varepsilon
\end{gather*}
\end{proof}
\begin{twr}[Kryterium Diniego]
Załóżmy, że $ f_n:\left[a,b\right] \to \mathbb R $ jest monotoniczny i zbieżny punktowo do funkcji $ f:\left[a,b\right] \to \mathbb R $. Jeśli $ f_n $ i $ f $ są ciągłe to zbieżność jest jednostajna.
\end{twr}
\begin{prz}
Brak ciągłości funkcji granicznej\qquad$ f_n(x)=x^n $ na $ \left[0,1\right] $\\
Brak monotoniczności\qquad
$ nxe^{-nx^2} $ na $ \left[0,2\right] $\\
Brak ciągłości funkcji\qquad$ \chi_{\left(0,\frac{1}{n}\right)} :[0,1]\to \mathbb R $\\
Brak zwartości dziedziny\qquad $ f_n(x)=x^n $ na $ [0,1) $
\end{prz}