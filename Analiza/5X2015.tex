\chapter{5 października 2015}
Szereg funkcyjny $ \sum_{n=1}^{\infty }f_n(x) $ to ciag sum częściowych
\begin{gather*}
S(_n(x)=f_1(x)+\dots+f_n(x)\\
s(x)=\lim\limits_{n\to\infty} s_n(x)
=
\lim\limits_{n\to\infty} \sum_{k=1}^{n}f_k(x)
=
\sum_{n=1}^{\infty }f_n(x)
\end{gather*}

\textbf{Wniosek}\\
Jeżeli funkcje $ f_n:[a,b]\to \mathbb R  $ są nieujemne, ciągłe i szereg $ \sum f_n(x) $ jest zbieżny punktowo do funkcji ciągłej to $ f_n $ jest zbieżny jednostajnie

\textsc{Kryterium Weierstrassa}\\
$ f_n:A\to \mathbb R \; \sum f_n(x) $. Jeśli $ u_n:=\sup_{x\in A}\left|f_n(x)\right| $ i $ \sum u_n $ zbieżny, to $ \sum f-n(x) $ jest zbieżny bezwzględnie i jednostajnie na $ A $.

\textsc{Warunek Cauchy'ego}\\
\begin{gather*}
\forall_{\varepsilon>0}\exists_{n_0}\forall_{n>k>n_0}
\left|f_k(x)+\dots+f_n(x)\right|<\varepsilon
\end{gather*}
warunek konieczny i dostateczny na zbieżność jednostajną $ \sum f_n(x) $
\begin{gather*}
\left|f_k(x)+\dots+f_n(x)\right|
\le
\left|f_k(x)\right|+\dots+\left|f_n(x)\right|
\le
u_k+\dots+u_n<\varepsilon
\end{gather*}

\textsc{Warunek Cauchy'ego dal ciagu}\\
$ a-n $ jest zbieżny wtedy i tylko wtedy, gdy
\begin{gather*}
\forall_{\varepsilon>0}\exists_{n_0}\forall_{n,k>n_0}\left|a_n-a_k\right|<\varepsilon
\end{gather*}
dla szeregu $ s_n=\sum_{k=1}^{n}a_k\\
s_n-s_k=a_{k+1}+\dots+a_n $
\begin{prz}
\begin{align*}
\sum_{n=0}^{\infty }\frac{x^n}{n!}&&\text{zbieżny jednostajnie na }&&[-\alpha,\alpha]\subset \mathbb R ,\;\alpha>0
\end{align*}
\begin{gather*}
\frac{x^n}{n!}\le \frac{\alpha^n}{n!}
\Rightarrow
\sum_{n=0}^{\infty }\frac{x^n}{n!}
\le
\sum_{n=0}^{\infty }\frac{\alpha^n}{n!}\\
\frac{a_{n+1}}{a_n}=\frac{\alpha}{n+1}\to 0\Rightarrow\text{ zbieżny}
\end{gather*}
\end{prz}
\begin{prz}
$ \sum_{n=1}^{\infty }e^{nx^2}n^2\,,x\in \mathbb R $\\
$ \sum_{n=1}^{\infty }\frac{\sin nx}{\sqrt[3]{n^4+x^2}}\qquad\left|\frac{\sin nx}{\sqrt[3]{n^4+x^2}}\right|\le \frac{1}{\sqrt[3]{n^4}} $ - zbieżność jednostajna, bezwzględna na $ \mathbb R  $

Przykłady szeregów zbieżnych, do których nie nadaje się kryterium Weierstrassa.
\begin{itemize}
	\item 
	\begin{gather*}
	\sum_{n=1}^{\infty }\frac{\sin nx}{n}
	\text{ zbieżność jednostajna na }A\subset [0,2\pi]
	\end{gather*}
	\item \begin{gather*}
	\sum_{n=1}^{\infty }f_n(x)\\
	f_n(x)=
	\left \{
	\begin{array}{ll}
		\frac{1}{x} & x\in[n,n+1) \\
		0           & x\notin[n,n+1)
	\end{array}
	\right .\\
	\sup_{x\ge 1}\left|f_n(x)\right|=\frac{1}{n}\\
	\sum_{k=1}^{n}f_k(x)=\frac{1}{x}\mathbbm1_{[q,n+1)}=
	\left \{
	\begin{array}{ll}
		\frac{1}{2} & x\in[1,n+1) \\
		0           & x\ge n+1
	\end{array}
	\right .\\
	\sup_{n\ge1}\left|s_n(x)-s(x)\right|=\sup_{x\ge1}
	\left \{
	\begin{array}{ll}
	\frac{1}{2} & x\in[1,n+1) \\
	0           & x\ge n+1
	\end{array}
	\right \}=\frac{1}{n+1}\to0	
	\end{gather*}
\end{itemize}
\end{prz}
\begin{twr}[Kryterium Abela]
	Załóżmy, że sumy częściowe $ \sum f_n(x) $ są jednostajnie ograniczone na zbiorze $ A $, tzn.
	\begin{gather*}
	\exists_{M>0}\forall_{n\in \mathbb N }\forall_{x\in A}\left|\sum_{k=1}^{n}f_k(x)\right|\le M
	\end{gather*}
	Jeśli $ a_n\to0 $ to szereg $ \sum a_nf_n(x) $ jest zbieżny jednostajnie na $ A $.
\end{twr}
\begin{prz}
	$ sum_{n=1}^{\infty }\frac{\sin nx}{n} $\\
	$ a_n\to0 $
	\begin{gather*}
	\exists_{M>0}\forall_{n\in \mathbb N }\forall_{x\in A} \left|\sum_{k=1}^{n}\sin kx\right|\le M
	\end{gather*}
\end{prz}