\chapter{5 Października 2015}
\begin{twr}
Niech $ \mathcal C $ będzie ciałem podzbiorów $ \Omega $ oraz $ \mu:\mathcal C\to\left[0,1\right] $ będzie miarą skończenie addytywną na $ \left(\Omega,\mathcal C\right) $ nieujemną i unormowaną, czyli 
\begin{gather*}
\mu(\Omega)=1,\forall_{a\in\mathcal C}\; 0\le\mu(A)\le 1.
\end{gather*} Wówczas następujące warunki są równoważne:
\begin{enumerate}[(A)]
\item $ \mu $ jest $\sigma$-addytywna na $ \left(a\Omega,\mathcal C\right) $
\item Ciągłość od dołu
\begin{gather*}
\forall_{B_j\in\mathcal C}\forall_{B_1\subseteq B_2\subseteq \dots} \bigcup_{j=1}^\infty B_j\in\mathcal C 
\Rightarrow
\mu \left(\bigcup_{j=1}^\infty B_j\right)=\lim\limits_{n\to\infty} \mu \left(B_j\right)
\end{gather*}
\item Ciągłość od góry
\begin{gather*}
\forall_{C_j\in\mathcal C}\forall_{C_1\supseteq C_2\supseteq \dots}
\bigcap_{j=1}^\infty C_j=\emptyset
\Rightarrow
\lim\limits_{n\to\infty} \mu \left(C_j\right)=0
\end{gather*}
\end{enumerate}
\end{twr}
\begin{proof}
(A) $ \Rightarrow $ (B)\\
$ A_1=B_1\\
A_2=B_2\backslash B_1\\
\vdots\\
A_n=B_n\backslash B_{n-1}\\
\vdots $\\
\begin{gather*}
\bigcup_{j=1}^\infty B_j=\bigcup_{j=1}^\infty A_j
\end{gather*}
\begin{align*}
&\mu \left(\bigcup_{j=1}^\infty B_j\right)
=
\mu \left(\bigcup_{j=1}^\infty A_j\right)
=\\=&
\sum_{j=1}^\infty \mu \left(A_j\right)
=
\lim\limits_{n\to\infty} \sum_{j=1}^n \mu \left(A_j\right)
=\\=&
\lim\limits_{n\to\infty} \mu \left(\bigcup_{j=1}^n A_j\right)
=\\=&
\lim\limits_{n\to\infty} \mu \left(B_n\right)
\end{align*}
(B) $ \Rightarrow $ (C)\\
$ C_1\supseteq C_2\supseteq\dots $\\
$ \bigcap_{n=1}^\infty C_n=\emptyset \\
B_n=C_n^\mathrm c=\Omega\backslash C_n\\
\bigcup_{n=1}^\infty B_n=\left(\bigcap_{n=1}^\infty C_n\right)^\mathrm c=\emptyset^\mathrm c=\Omega$
\begin{align*}
&1=\mu \left(\Omega\right)
=\\=&
\mu \left(\bigcup_{n=1}^\infty B_n\right)
=\\=&
\lim\limits_{n\to\infty} \mu \left(B_n\right)
=\\=&
\lim\limits_{n\to\infty} \left(\Omega\backslash C_n\right)
=\\=&
\lim\limits_{n\to\infty} \bigl(\mu\left(\Omega\right)-\mu\left( C_n\right)\bigr)
=\\=&
\mu\left(\Omega\right)-\lim\limits_{n\to\infty} \mu\left( C_n\right)
\Rightarrow
\lim\limits_{n\to\infty} \mu \left(C_n\right)=0
\end{align*}
(C) $ \Rightarrow $ (A)\\
$ \bigcup_{j=1}^\infty A_j \in\mathcal C\\
\bigcup_{j=1}^{n-1} A_j \in\mathcal C$
\begin{gather*}
C_n
=
\bigcup_{j=n}^\infty A_j
=
\bigcup_{j=1}^\infty A_j\backslash\bigcup_{j=1}^{n-1} A_j\in\mathcal C
\end{gather*}
$ C_1\supseteq C_2\supseteq\dots \\
\bigcap_{j=1}^\infty C_n = \emptyset$.\\
Z (C) $ \lim\limits_{n\to\infty} \mu \left(C_n\right)=0 $
\begin{align*}
&\mu\left(C_n\right)
=\\=&
\mu \left(\bigcup_{j=n}^\infty A_j\right)
=\\=&
\mu\left(\bigcup_{j=1}^\infty A_j\backslash\bigcup_{j=1}^{n-1} A_j\right)
=\\=&
\mu\left(\bigcup_{j=1}^\infty A_j\right)-\mu\left(\bigcup_{j=1}^{n-1} A_j\right)
=\\=&
\mu\left(\bigcup_{j=1}^\infty A_j\right)-\sum_{j=1}^{n-1} \mu\left(A_j\right)\to0
\end{align*}
\begin{gather*}
\lim\limits_{n\to\infty} \mu\left(\bigcup_{j=1}^{n-1} A_j\right)=\mu\left(\bigcup_{j=1}^{\infty } A_j\right)\\
\lim\limits_{n\to\infty} \sum_{j=1}^{n-1} \mu\left(A_j\right)=\mu\left(\bigcup_{j=1}^{\infty } A_j\right)
\end{gather*}
Ostatecznie
\begin{gather*}
\mu\left(\bigcup_{j=1}^{\infty } A_j\right)=
\sum_{j=1}^{\infty } \mu\left(A_j\right)
\end{gather*}
\end{proof}
\textbf{Uwaga!}\\
Warunek (C) jest bardzo wygodny do sprawdzenia.
\begin{twr}
Jeżeli $ \mu $ jest prawdopodobieństwem $\sigma$-addytywnym na ciele $ \mathcal C $ podzbiorów $ \Omega $, to istnieje dokładnie jedno rozszerzenie $ \mu  $ do miary probabilistycznej $ \tilde{\mu} $ na $ \sigma\left(\mathcal C\right) \left\{\tilde{\mu}|_\mathcal C=\mu\right\}$
\begin{gather*}
\left(\Omega,\mathcal C,\mu\right)
\rightsquigarrow
\left(\Omega,\sigma\left(\mathcal C\right),\tilde{\mu}\right)
\end{gather*}
Tradycyjnie $ \tilde{\mu} $ piszemy $ \mu $.
\end{twr}
\textbf{Dygresja}\\
$ \Omega=\mathbb N ,A\subseteq \mathbb N ,\nu \left(A\right) =\lim\limits_{n\to\infty} \dfrac{\mu \left(A\cap\left\{1,2,\dots,n\right\}\right)}{n}$ gęstość zbioru $ A $.
Dobra miara "grubości" podzbiorów $ \mathbb N  $. Nie wszystkie podzbiory mają gęstość. Z całą pewnością $ \nu$ nie jest $\sigma$-addytywna, bo $ \nu(k)=0,\,\text{card}(k)<\infty $\\
\begin{gather*}
1=\nu\left(\mathbb N \right) \neq \lim\limits_{n\to\infty} \nu \left(\left\{1,2,\dots,n\right\}\right)=0
\end{gather*}
\begin{prz}
$ \Omega=(0,1]\\
\mathcal{C_{\text{przed.}}}=\left\{\bigcup_{j=1}^\infty (\alpha_j,\beta_j]:n=0,1,\dots;0<\alpha_1<\beta_1<\alpha_2<\beta_2<\dots<\alpha_n<\beta_n\right\}\\
F:[0,1]\to\mathbb R,F $ niemalejące, prawostronnie ciągłe.
\begin{gather*}
\mu_F \left(\bigcup_{j=1}^\infty (\alpha_j,\beta_j]\right)\stackrel{df}{=}
\sum_{j=1}^{n}\bigl(F\left(B_j\right)-F\left(\alpha_j\right)\bigr)\ge0
\end{gather*}
$ \mu_F $ jest skończenie addytywna na $ \left((0,1],\mathcal{_{\text{przed.}}}\right) $, co widać z konstrukcji. Nieco trudniej dowodzi się, że $ \mu_F $ jest $\sigma$-addytywna na ciele $ \mathcal C_{\text{przed.}} $. Zatem każda funkcja niemalejąca, prawostronnie ciągła $ F:\left[0,1\right] $ definiuje miarę $\sigma$-addytywną na $ \left((0,1],\sigma\left(\mathcal C_{\text{przed.}}\right)\right) $
\end{prz}
\begin{center}
{\LARGE \textsc{Funkcje tworzące}}\\
(p.g.f. probability geometry function)
\end{center}
Niech $ X:(\Omega,\mathcal F,P)\to \mathbb N _0=\left\{0,1,2,\dots\right\} $ będzie zmienną losową ($ ychX^{-1}(B)\in\ \mathcal F,\subseteq \mathbb N $ - dowolny podzbiór) . Nową charakterystyką takich zmiennych losowych jest funkcja charakterystyczna.
\begin{defi}[Funkcja tworząca]Funkcją tworzącą zmiennej losowej $ X $ o wartościach w $ \mathbb N _0 $ nazywamy
\begin{gather*}
\Upsilon_X(s)=\mathbb E s^X\left(=\sum_{j=0}^{\infty }s^jP(X=j)=\sum_{j=0}^{\infty }s^jp_j\right)
\end{gather*}
\end{defi}
Pytanie: Dom$ (r_X) $=?\\
$ r_X $ określona szeregiem potęgowym; pewnie analityczna. $ r_X(z),z\in K(0,R) $
\begin{twr}
Niech $ X.\tau,X^{(1)},X^{(2)},\dots $ będą zmiennymi losowymi losowymi o wartościach w $ \mathbb N _0 $ i oznaczmy $ P\left(X=k\right)=p_k $ , $ P\left(X^(n)=k\right)=p_k^{(n)} ,\,k=0,1,2,\dots$. Wówczas:
\begin{enumerate}
\item dom$ \left(\Upsilon_X\right) \supseteq[-1,1]$
(W przypadku dziedziny zespolonej $ \overline K(0,1) $)\\
$ \Upsilon_X $ jest niemalejąca i wypukła na $ [0,1] $oraz klasy $ C^\infty  $ na $ (-1,1) $
\item $ \Upsilon_X(1)=1 $
\item \begin{align*}
\Upsilon_X'&(0)= P\left(X=1\right)=p_1        \\
\Upsilon_X''&(0)=2\cdot P\left(X=2\right)=2p_2\\
&\vdots                                       \\
\Upsilon_X^{(k)}&(0)=k!\cdot P\left(X=k\right)=k!p_k
\end{align*}
\item Dla $ X $
\begin{gather*}
\mathbb E X^n<\infty 
\Leftrightarrow
\lim\limits_{n\to1^-} \Upsilon_X^{(n)}(s)
=
\Upsilon_X^{(n)}(1^-)<\infty 
\end{gather*}
Co więcej
\begin{gather*}
\Upsilon_X^{(n)}(1^-)
=
\mathbb E X(X-1)\cdots(X-n+1)\\
\text{  n-ty moment faktorialny}
\end{gather*}
W szczególności
\begin{gather*}
\mathbb E X
=
\Upsilon_X'(1^-)\\
\mathbb E X^2
=
\mathbb E X(X-1)+\mathbb E X
=
\Upsilon_X''(1^-)+\Upsilon_X'(1^-)
\end{gather*}
\item Jeżeli $ x $ i $ Y $ są niezależne o wartościach w $ \mathbb N _0 $, to $ \Upsilon_{X+Y}(s)=\Upsilon_X(s)\Upsilon_Y(s) $
\item Jeżeli $ X^{(1)},X^{(2)},\dots $ są niezależnymi zmiennymi losowymi o tym samym rozkładzie i o wartościach w $ \mathbb N _0 $ i podobnie $ \tau $ i dodatkowo ciąg $ X^{(1)},X^{(2)},\dots $ i $ \tau $ są niezależne, to dla 
\begin{gather*}
U=\sum_{j=1}^{\tau}X^{(j)}
\end{gather*}
mamy
\begin{gather*}
\Upsilon_U(s)=\Upsilon_\tau \bigl(\Upsilon_{X^{(1)}}(s)\bigr)
\end{gather*}
\item Dla dowolnego ciągu zmiennych losowych (niekoniecznie niezależnych) $ X^{(1)},X^{(2)},\dots $ i zmiennej losowej $ X $ o wartościach w $ \mathbb N _0 $ następujące warunki są równoważne:
\begin{enumerate}
\item $ \forall_{k\in \mathbb N _0}\lim\limits_{n\to\infty} P\left(X^{(n)}=k\right)=P\left(X=k\right) $
\item $ X_n\overset{\mathcal D}{\Rightarrow}X $ (zbieżność słaba)
\item $ \forall_{s\in[0,1]}\lim\limits_{n\to\infty} \Upsilon_{X^{(n)}}(s)=\Upsilon_X(s) $
\end{enumerate}
\end{enumerate}
\end{twr}
\begin{proof}\text{ }
\begin{itemize}
\item[2.] $ \Upsilon_X(1)=\mathbb E 1^X=\mathbb E 1=1 $
\item [1.]
\begin{gather*}
\Upsilon_X(s)
=
\sum_{k=0}^{\infty }s^Xp_k
=
\sum_{k=0}^{\infty }p_ks^k
\end{gather*}
ten szereg potęgowy jest zbieżny dla $ s=1 $ nawet bezwzględnie.
\begin{align*}
&\left(\sum_{k=0}^{\infty }\left|p_k\right|\left|s\right|^k \right) _{s=1}\\
&\sum_{k=0}^{\infty }\left|a_k\right|\left|z^k\right|<\infty 
\end{align*}
$ \Upsilon_X(z) $ jest dobrze określone na $ \left|z\right| \le1$.\\
$ \Upsilon_X $ jest funkcją analityczną (co najmniej) na $ \left\{z:\left|z\right|<1\right\} $, a stąd wynika, że $ \Upsilon_X $ jest ciągła na $ [-1,1] $ i ma ciągłe pochodne na $ (-1,1) $.\\
$ \Upsilon_X(s)=\sum_{k=0}^{\infty }p_k s^k\ge0$ na $ [0,1] $\\
$ \Upsilon'_X(s)=\sum_{k=0}^{\infty }k\cdot p_k s^{k-1}\ge0$ - niemalejąca na $ [0,1] $\\
$ \Upsilon''_X(s)=\sum_{k=0}^{\infty }k(k-1) p_k s^{k-1}\ge0 $ dla $ s\in[0,1] $\\
Reasumując $ \Upsilon_X $ jest nieujemna, niemalejąca, wypukła na $ [0,1] $.\\
\begin{align*}
&P\left(\left\{X+0\right\}\cup\left\{X=1\right\}\right)=1
&&\Upsilon_X(s)=p_0+p_1\cdot s\\
&P\left(X\ge2 \right)>0
&&\Upsilon_X \text{ ściśle wypukła}
\end{align*}
\item [3.]
\begin{gather*}
\Upsilon_X^{(k)}(s)=\frac{d^k}{ds^k}\left(\sum_{j=0}^{\infty }p_js^j\right)
=\\=
\sum_{j=0}^{\infty }j(j-1)\cdots(j-k+1)p_js^{j-k}
=
\sum_{j=k}^{\infty }j(j-1)\cdots(j-k+1)p_js^{j-k}
\end{gather*}
\begin{gather*}
\Upsilon_X^{(k)}(0)=k(k-1)\cdots(k-k+1)p_k=k!p_k\Rightarrow p_k=P\left(X=k\right)=\frac{\Upsilon_X}{k!}
\end{gather*}
\item [4.]
\begin{align*}
&\mathbb E X
=\\=&
\sum_{k=0}^{\infty }k\cdot P\left(X=k\right)
=\\=&
\sum_{k=0}^{\infty }k\cdot p_k\cdot 1^k
=\\=&
\left(\sum_{k=0}^{\infty }k\cdot p_k\cdot s^k\right)_{s=1}
=\\=&
\lim\limits_{n\to1^-} \left(\sum_{k=0}^{\infty }k\cdot p_k\cdot s^k\right)
=\\=&
  \lim\limits_{n\to1^-} \Upsilon_X'(s)
\end{align*}
Analogicznie
\begin{align*}
&\mathbb E X(X+1)\cdots(X-n+1)
=\\=&
\sum_{k=n}^{\infty }k(k-1)\cdots(k-n+1)p_k\cdot 1^{k-n+1}
=\\=&
\lim\limits_{n\to1^-} \Upsilon_X^{(n)}(s)
\end{align*}
\textbf{Uwaga!}\\
\begin{gather*}
\mathbb E X^n<\infty \Leftrightarrow\mathbb E (X(X-1)\cdots(X-n+1)<\infty 
\end{gather*}
\item [5.]
$ X,Y $ niezależne zmienne losowe o wartościach w $ \mathbb N _0 $
\begin{gather*}
\Upsilon_{X+Y}(s)
=
\mathbb E s^{X+Y}
=
\mathbb E s^X\cdot s^Y
=
\mathbb E s^X\cdot \mathbb E s^Y
=
\Upsilon_X^{(s)}\Upsilon_Y^{(s)}
\end{gather*}
\item [6.] $ V=\sum_{j=1}^{\tau}X^{(j)} $, ustalmy $ \sum_{j=1}^{0}\dots=0 $
\begin{align*}
&\Upsilon_V(s)=\mathbb E s^V
=
\int\limits_{\Omega}s^V\,dP
=\\=&
\sum_{n=0}^{\infty }\int\limits_{\left\{\tau=n\right\}}s^{\sum_{j=1}^{\tau}x^{(j)}}\,dP
=\\=&
\int\limits_{\left\{\tau=0\right\}}s^{\sum_{j=1}^{0}x^{(j)}}\,dP+
\sum_{n=1}^{\infty }\int\limits_{\left\{\tau=n\right\}}s^{\sum_{j=1}^{\tau}x^{(j)}}\,dP
=\\=&
1\cdot P\left(\tau=0\right)+
\sum_{n=1}^{\infty }
\int\limits_{\Omega}\mathbbm1_{\left\{\tau=n\right\}}\,dP\int\limits_{\Omega}s^{\sum_{j=1}^{\tau}x^{(j)}}\,dP
=\\=&
1^0\cdot P\left(\tau=0\right)+\sum_{n=1}^{\infty }P\left(\tau=n\right)\Upsilon_{\sum_{j=1}^{\tau}x^{(j)}}(s)
=\\=&
P\left(\tau=0\right)\cdot s^0+\sum_{n=1}^{\infty }P\left(\tau=n\right)\left[\Upsilon_{X^{(1)}}(s)\right]^n
=\\=&
P\left(\tau=0\right)\cdot \left(\Upsilon_X^{(1)}(s)\right)^0+\sum_{n=1}^{\infty }P\left(\tau=n\right)\left[\Upsilon_{X^{(1)}}(s)\right]^n
=\\=&
\Upsilon_\tau \bigl(\Upsilon_{X^{(1)}}(s)\bigr)
=
\Upsilon_\tau \circ \Upsilon_{X^{(1)}}(s)
\end{align*}
\end{itemize}
\end{proof}