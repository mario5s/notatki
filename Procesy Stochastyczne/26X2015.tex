\chapter{26 października 2015}
\begin{defi}[Proces Poissona]
Mówimy, że proces $ \left\{N_t\right\}_{t\in[0,\infty )} $ określony na przestrzeni probabilistycznej $(\Omega,\mathcal F,P)$ jest procesem liczącym, jeżeli
\begin{enumerate}
\item $ N_0=0 $
\begin{gather*}
\forall_{\omega\in\Omega}\forall_{t\in T}N_T(\omega)\in \mathbb N _0=\left\{0,1,2,\dots \right\}
\end{gather*}
\item 
\begin{gather*}
\forall_{0\le s<t}\forall_{\omega\in\Omega}\;
N_s(\omega)\le N_t(\omega)
\end{gather*}
\item $ \forall_{0\le s<t}\; N_t(\omega)-N_s(\omega) $ reprezentuje liczbę zdarzeń jakie zaszły na odcinku czasowym $ (s,t] $.
\end{enumerate}
\end{defi}
\begin{defi}[Proces jednorodny]
Mówimy, że proces liczący $ \left\{N_t\right\}_{t\in[0,\infty )} $ określony na   przestrzeni probabilistycznej $(\Omega,\mathcal F,P)$jest jednorodnym procesem Poissona z intensywnością $ \lambda>0 $ - liczba, jeżeli spełnia
\begin{enumerate}[i]
\item $ N(0)=0 $ z prawdopodobieństwem 1
\item $ \left\{N_t\right\}_{t\in[0,\infty )} $ ma przyrosty niezależne
\item $ \forall_{0\le s<t}\; N_t(\omega)-N_s(\omega) $ ma rozkład Poissona z parametrem $ \lambda(t-s) $
\begin{gather*}
\left[\forall_{k\in\left\{0,1,2,\dots\right\}}\;
P\bigl(N(t)-N(s)=k\bigr)=\frac{\left[\lambda(t-s)\right]^k}{k~}e^{-\lambda(t-s)}
\right]
\end{gather*}
\end{enumerate}
\end{defi}

\textbf{Uwaga!}\\
(iii) implikuje, że przyrosty są jednorodne
\begin{gather*}
N(t+h)-N(s+h)\sim
Poiss\bigl(\lambda(t+h-(s+h))\bigr)=
Poiss\bigl(\lambda(t-s)\bigr)\sim
N(t)-N(s)
\end{gather*}

Zatem (iii) w połączeniu z (ii) mamy proces o przyrostach stacjonarnych (niezależnych).

\textbf{Uwaga!}\\
\begin{gather*}
\mathbb E N(t)=\lambda t\\
\Var N(t)=\lambda t
\end{gather*}
\begin{twr}
Niech $ \left\{N_t\right\}_{t\in[0,\infty )} $ będzie procesem liczącym określonym na przestrzeni probabilistycznej $(\Omega,\mathcal F,P)$. Wówczas $ \left\{N_t\right\}_{t\in[0,\infty )} $ jest jednorodnym procesem Poissona z intensywnością $ \lambda>0 $ wtedy i tylko wtedy, gdy spełnia\begin{enumerate}
\item $ N(0)=0 $ z pr. 1
\item $ \left\{N_t\right\}_{t\in[0,\infty )} $ ma jednorodne i niezależne przyrosty
\item 
\begin{gather*}
 P\bigl(N(h)=1\bigr)=\lambda\cdot h+o(h) \\
 \left[\frac{o(h)}{h}\xrightarrow[h\to0]{}0\right]
\end{gather*}
\item 
\begin{gather*}
P\bigl(N(h)\ge2 \bigr)=P\bigl(N(h)=2\vee N(h)=3\vee\dots\bigr)=o(h)
\end{gather*}
\end{enumerate}
\end{twr}
\begin{proof}
{\Large $ \Rightarrow $}\\
(i)$ \equiv $(1)\\
$ (ii)+(iii)\Rightarrow(2) $\\
\begin{gather*}
P\bigl(N(h)=1\bigr)\stackrel{(iii)}{=}
\frac{(\lambda h)^1}{1!}e^{-\lambda h}=
\lambda he^{-\lambda h}
\end{gather*}
3)
\begin{align*}
&P\bigl(N(h)=1\bigr)-\lambda h\stackrel{?}{=}o(h)\\
&\frac{\lambda he^{-\lambda h}-\lambda h}{h}=
\lambda  \left(e^{-\lambda h}-1\right)\xrightarrow[h\to0]{}
\left(e^{-\lambda 0}-1\right)=0
\end{align*}
4)
\begin{align*}
&\frac{P\bigl(N(h)\le 2\bigr)}{h}\xrightarrow[h\to0]{?}0\\
&\frac{1-P\bigl(N(h)=0\bigr)-P\bigl(N(h)=1\bigr)}{h}\stackrel{(iii)}{=}
\frac{1-e^{-\lambda h}-\lambda he^{-\lambda h}}{h}
\end{align*}
Stosując regułę de l'Hospitala
\begin{gather*}
\lim\limits_{h\to0} \frac{P\bigl(N(h)\le 2\bigr)}{h}=
\lim\limits_{h\to0} \frac{\lambda  e^{-\lambda h }-\lambda  \left(e^{-\lambda h }-\lambda h  e^{-\lambda h }\right)}{1}=
\lim\limits_{h\to0} \lambda ^2 h e^{-\lambda h }=0
\end{gather*}
{\Large $ \Leftarrow $}\\
(i)$ \equiv $(1)\\
(2)$ \Rightarrow $(ii)\\
Jak pokazać (iii)? $ P_n(t)=P\bigl(N(t)=n\bigr) \qquad n=0,1,2,\dots $

Etapami
\begin{gather*}
P_0(t)=P\bigl(N(t)=0\bigr)\stackrel{?}{=}e^{-\lambda h}=\frac{(\lambda
 h)^0}{0!}e^{\lambda h}
\end{gather*}
\begin{align*}
P_0(t+h)=&
P\bigl(N(t+h)=0\bigr)
=\\=&
P\bigl(N(t)=0\wedge N(t+h)-N(t)=0\bigr)
\stackrel{\mathrlap{N(t)\Perp N(t+h)}}{=}\\=&
P\bigl(N(t)=0\bigr) P\bigl(N(t+h)-N(t)=0\bigr)
\stackrel{(2)}{=}\\=&
P_0\left(t\right)\cdot P\bigl(N(h)=0\bigr)
\stackrel{(3)}{=}\\=&
P_0\left(t\right)\cdot \Bigl(1-P\bigl(N(h)=1\bigr)-P\bigl(N(h)\ge 2\bigr)\Bigr)
=\\=&
P_0(t)\left(1-\lambda h-o(h)-o(h)\right)
=\\=&
P_0(t)-\lambda  P_0(t)h -2 o(h) P_0(t)
=\\=&
P_0(t)-\lambda  P_0(t)h - o(h)
\end{align*}
\begin{gather*}
\frac{P_0(t+h)-P_0(t)}{h}=
-\lambda P_0(t)- \frac{o(h)}{h}\\
P_0'(t)=-\lambda P_0(t)
\end{gather*}
z warunkiem $ P_0(0)=P\bigl(N(0)=0\bigr)=1 $\\
$ P_0(t)=e^{\lambda t} \\
P_0(0)=P\bigl(N(0)=0\bigr)=
\frac{(\lambda h)^0}{0!}e^{\lambda h}$
 
Indukcyjnie pokazuje się, ze
\begin{gather*}
\forall_{n\ge0}\;
P\bigl(N(t)=n\bigr)=
\frac{(\lambda h)^n}{n!}e^{\lambda h}
\end{gather*}
Pokazaliśmy, już I krok indukcyjny i formuła zachodzi dla n=0.
\begin{align*}
P_n(t+h)=&
P\bigl(N(t+h)=n\bigr)
=\\=&
P\bigl(
\left\{N(t)=n\wedge N(t+h)-N(t)=0\right\}
\dot{\cup}\\&\dot{\cup}
\left\{N(t)=n-1\wedge N(t+h)-N(t)=1\right\}
\dot{\cup}\\&\dot{\cup}
\left\{N(t)<n-1\wedge N(t+h)-N(t)\ge2\right\}
\bigr)
=\\=&
P\bigl(N(t)=n\wedge N(t+h)-N(t)=0\bigr)
+\\&+
P\bigl(N(t)=n-1\wedge N(t+h)-N(t)=1\bigr)
+\\&+
P\bigl(N(t)<n-1\wedge N(t+h)-N(t)\ge2\bigr)
\stackrel{(2)}{=}\\=&
P\bigl(N(t)=n\bigr) P\bigl( N(h)=0\bigr)
+\\&+
P\bigl(N(t)=n-1\bigr) P\bigl( N(h)=1\bigr)
+\\&+
P\bigl(N(t)<n-1\bigr) P\bigl(N(h)\ge2\bigr)
=\\=&
P\bigl(N(t)=n\bigr) P\bigl( N(h)=0\bigr)
+
P\bigl(N(t)=n-1\bigr) P\bigl( N(h)=1\bigr)
+
o(h)
=\\=&
P_n(t)e^{-\lambda h}+P_{n-1}(t)\bigl(\lambda h+o(h)\bigr)+o(h)
=\\=&
P_n(t)\bigl(1-\lambda h+o(h)\bigr)+\lambda hP_{n-1}(t)+o(h)
=\\=&
P_n(t)-\lambda hP_n(t)+\lambda hP_{n-1}+o(h)
\end{align*}
\begin{gather*}
P_n(t+h)=P_n(t)-\lambda hP_n(t)+\lambda hP_{n-1}+o(h)\\
\frac{P_n(t+h)-P_n(t)}{h}=\frac{-\lambda hP_n(t)+\lambda hP_{n-1}(t)}{h}+\frac{o(h)}{h}
\end{gather*}
Przechodząc $ h\to0 $
\begin{align*}
&P_n'(t)=-\lambda P_n(t)+\lambda P_{n-1}(t)\\
&P_n'(t)+\lambda P_n(t)=\lambda P_{n-1}(t)\\
&P_n'(t)e^{\lambda t}+\lambda P_n(t)e^{\lambda t}=\lambda P_{n-1}(t)e^{\lambda t}\\
&\left(P_n(t)e^{\lambda t}\right)'=\lambda P_{n-1}(t)e^{\lambda t}\qquad n=1,2,3,\dots
\end{align*} 
dla $ n=1 $
\begin{gather*}
\left(P_1(t)e^{\lambda t}\right)'=\lambda P_{0}(t)e^{\lambda t}=\lambda e^{-\lambda t} e^{\lambda t}=\lambda\\
P_1(t) e^{\lambda t}=\lambda t+C
\end{gather*}
Warunek początkowy
\begin{gather*}
P_1(0)=P\bigl(N(0)=1\bigr)=0
\end{gather*}
daje
\begin{gather*}
P_1(0)e^{\lambda 0}=\lambda\cdot 0 +C\Rightarrow C=0
\end{gather*}
Ostatecznie
\begin{gather*}
P_1(t) e^{\lambda t}=\lambda t\\
P_1(t)=\frac{\left(\lambda t\right)^1}{1!} e^{-\lambda t}
\end{gather*}
W II kroku indukcyjnym załóżmy, że
\begin{align*}
&\left(P_n\left(t\right)e^{\lambda t}\right)'=
\lambda\frac{\left(\lambda t\right)^{n-1}}{(n-1)!}e^{-\lambda t}\cdot e^{\lambda t}=
\frac{\lambda^n\cdot  t^{n-1}}{(n-1)!}\\
&P_n\left(t\right)e^{\lambda t}=
\int \frac{\lambda ^n t^{n-1}}{(n-1)!} \, dt=
\frac{\lambda ^n}{n!} ^n+C
\end{align*}
Warunek
\begin{gather*}
P_n(0)=
P\left(N(0)=n\right)=0\Rightarrow C=0
\end{gather*}
Na mocy indukcji matematycznej
\begin{gather*}
\forall_{n\ge 0}P\left(N(t)=n\right)=P_n(t)=\frac{(\lambda t)^n}{n!}e^{-\lambda t}
\end{gather*}
(iii)
\begin{align*}
& P\left(N(t)-N(s)=n\right)
=\\=&
P\left(N(s+(t-s))-N(s)=n\right)
\stackrel{(2)}{=}\\=&
P\left(N(t-s)=n\right)=
\frac{(\lambda(t-s))^n}{n!}e^{-\lambda(t-s)}
\end{align*}
\end{proof}
\section{Własności trajektorii Procesu Poissona}
\textbf{Fakt 1}\\
Z prawdopodobieństwem 1 trajektoria ma tylko skończenie wiele skoków na każdym skończonym odcinku czasowym $ [0,t] $.\\
$ A_n $ - zdarzenie polegające na tym, że na odcinku $ [0,n] $ było nieskończenie wiele skoków\\
$ A_n $ - zdarzenie polegające na tym, że na odcinku skończonym było skończenie wiele skoków
\begin{gather*}
B^\mathrm c=\Omega\backslash B=\bigcup_{n=1}^\infty A_n
\end{gather*}
\begin{align*}
&P\left(A_n\right)
=\\=&
P\left(\text{na odcinku }[0,n]\text{ było nieskończenie wiele skoków}\right)
=\\=&
P\left(\forall_{k\in \mathbb N }N(n)\ge k\right)
=\\=&
P\left(\bigcap_{k=1}^\infty \left\{N(n)\ge k\right\}\right)
=\\=&
\lim\limits_{k\to\infty} P\left(N(n)\ge k\right)
=\\=&
\lim\limits_{k\to\infty} P\left(\bigcup_{j=k}^\infty N(n)=j\right)
=\\=&
\lim\limits_{k\to\infty} \underbrace{\sum_{j=k}^\infty \frac{(\lambda n)^j}{j!}e^{-\lambda n}}_{\text{ogon szeregu zbieżnego}}=0
\end{align*}
\begin{gather*}
\forall_nP\left(A_n\right)=0
\end{gather*}
\begin{align*}
&0\le P\left(\bigcup_{n=1}^\infty A_n\right)\le \sum_{n=1}^{\infty }P\left(A_n\right)=0\\
&P\left(B^\mathrm c\right)=0\Rightarrow P\left(B\right)=1-P\left(B^\mathrm c\right)=1-0=1
\end{align*}
\textbf{Uwaga!}\\
Jednorodny proces Poissona nie eksploduje (na skończonym odcinku czasu).\\
\textbf{Fakt 2}\\
Z prawdopodobieństwem 1 skoki trajektorii są równe 1.
\begin{gather*}
P\left(\left\{\omega\in\Omega:[0,\infty )\ni t\to N_t(\omega)\text{ ma skoki równe }1\right\}\right)=1
\end{gather*}
Zdarzenie przeciwne
\begin{align*}
&\left\{\omega\in\Omega:[0,\infty )\ni t\to N_t(\omega)\text{ ma skoki }\ge 2\right\}
=\\=&
\bigcup_{n=1}^\infty \left\{\omega\in\Omega:[0,\infty )\ni t\to N_t(\omega)\text{ ma skoki }\ge 2\text{ na odcinku }[0,n]\right\}=
\bigcup_{n=1}^\infty A_n
\end{align*}
\begin{gather*}
A_n=\bigcup_{j=0}^{n\cdot m-1}
\left\{\text{skok $ \ge 2 $ zdarzył się na }(\tfrac{j}{m},\tfrac{j+1}{m}]\right\}\le
\bigcup_{j=0}^{n\cdot m-1}
\left\{N\left(\frac{j+1}{m}\right)-N\left(\frac{j}{m}\right)\ge 2\right\}
\end{gather*}
\begin{align*}
&0\le P\left(A_n\right)\le
\sum_{j=0}^{n\cdot m -1}P\Bigl(N\left(\tfrac{j}{m}+\tfrac{1}{m}\right)-N\left(\tfrac{j}{m}\right)\ge 2\Bigr)
\underset{\text{jednorodność}}{\stackrel{\text{\text{przyrostu}}}{=}}\\
=
&\sum_{j=0}^{n\cdot m -1}
P\Bigl(N\left(\tfrac{1}{m}\right)\ge 2\Bigr)
=
n\cdot m\Bigl(1-P\left(N\left(\tfrac{1}{m}\right)=0\right)-P\left(N\left(\tfrac{1}{m}\right)=1\right)\Bigr)
=\\=&
n\left(1-e^{-\frac{1}{m}\cdot\lambda}-\frac{1}{m}e^{-\frac{1}{m}\cdot\lambda}\right)\cdot \frac{1}{\frac{1}{m}}
\le n\cdot\varepsilon
\end{align*}
\begin{gather*}
\lim\limits_{x\to0^+}
\frac{1-e^{-\lambda x}-\lambda xe^{-\lambda x}}{x}
\stackrel{\text{"H"}}{=}
\lim\limits_{x\to0^+}
\frac{\lambda e^{-\lambda  x}-\lambda e^{-\lambda  x}+\lambda ^2 x e^{-\lambda  x}}{1}=0
\end{gather*}
Jeśli
\begin{gather*}
X<\delta_\varepsilon
\end{gather*}
wtedy
\begin{gather*}
\left|\frac{1-e^{-\lambda x}-\lambda xe^{-\lambda x}}{x}\right|<\varepsilon
\text{ dla }\frac{1}{m}<\delta_\varepsilon\Bigl(\equiv\left(m>\tfrac{1}{\delta_\varepsilon}\right)\Bigr)
\end{gather*}
\begin{gather*}
0\le P\left(A_n\right)\le\varepsilon\cdot n
\end{gather*}
Zatem $ \forall_n P\left(A_n\right)=0 $\\
Stąd \begin{gather*}
P\left(\bigcup_{n=1}^\infty A_n\right)=0
\end{gather*}
\begin{align*}
&P\left(\left\{\omega\in\Omega:[0,\infty )\ni t\to N_t(\omega)\text{ ma skok }\ge 2\right\}\right)=0\\
&P\left(\left\{\omega\in\Omega:[0,\infty )\ni t\to N_t(\omega)\text{ ma skok }=1\right\}\right)=1
\end{align*}
\textbf{Fakt 3}\\
Z prawdopodobieństwem 1 trajektorie dążą do $ \infty  $, gdy $ t\to\infty  $
\begin{gather*}
P\left(\left\{\omega\in\Omega:\lim\limits_{t\to\infty} N_t(\omega)=\infty \right\}\right)=1
\end{gather*}
Zdanie przeciwne
\begin{gather*}
\left\{\omega\in\Omega:\lim\limits_{t\to\infty} N_t(\omega)<\infty \right\}
=
\bigcup_{n=1}^{\infty }
\left\{\omega\in\Omega:\lim\limits_{t\to\infty} N_t(\omega)<n \right\}
=
\bigcup_{n=1}^{\infty }A_n
\end{gather*}
\begin{align*}
A_n=&
\left\{\omega\in\Omega:\forall_{t\in[0,\infty )} N_t(\omega)\le n \right\}
=\\=&
\left\{\omega\in\Omega:\forall_{k\in \mathbb N } N_k(\omega)\le n \right\}
=\\=&
\bigcap_{k=1}^{\infty }
\left\{\omega\in\Omega: N_k(\omega)\le n \right\}
\end{align*}
\begin{align*}
P\left(A_n\right)=&
P\left(\bigcap_{k=1}^{\infty }\left\{N_k\le n\right\}\right)
=\\=&
\lim\limits_{k\to\infty}
P\left(N_k\le n\right)
=\\=&
\lim\limits_{k\to\infty}
P\left(N_k=0\vee N_k=1\vee\dots\vee N_k=n\right)
=\\=&
\lim\limits_{k\to\infty}
\sum_{j=0}^{n}P\left(N_k=j\right)
=\\=&
\lim\limits_{k\to\infty}
\sum_{j=0}^{n} \frac{(\lambda k)^j}{j!}\cdot e^{-\lambda k}=0
\end{align*}
\begin{gather*}
P\left(\bigcup_{n=1}^{\infty }A_n\right)=0
\end{gather*}
Stąd z prawdopodobieństwem 1
\begin{gather*}
\lim\limits_{t\to\infty} N_t=\infty 
\end{gather*}
\begin{twr}
Jeżeli $ \left\{N_t\right\}_{t\in[0,\infty )} $ jest jednorodnym procesem Poissona z intensywnością $ \lambda>0 $, to z prawdopodobieństwem 1 trajektorie $ [0,\infty )\ni t\to N_t(\omega) $ są funkcjami schodkowymi, startującymi z 0, o skokach równych 1, o skończenie wielu skokach na każdym odcinku skończonym i dążących do nieskończoności (gdy $ t\to\infty $) i prawostronnie ciągłymi.
\end{twr}\noindent
Moment skoku $ \equiv $ chwila skoku $ \left(T_1,T_2,\dots \right) $\\
Czas oczekiwania na kolejny skok $ X_1,X_2,\dots  $ "międzyczasy"; "czasypomiędzy"
\begin{twr}
Ciąg zmiennych losowych $ X_1,X_2,\dots$ dla jednorodnego procesu Poissona z intensywnością $ \lambda>0 $ jest ciągiem niezależnym o tym samym rozkładzie Exp$ \left(\lambda\right) $.
\begin{proof}
\begin{gather*}
P\left(X_1\le t\right)
=
P\left(N_t\ge 1\right)
=
1-P\left(N_t=0\right)
=
1-e^{-\lambda t}\qquad\text{dla }t\ge 0
\end{gather*}
$ F_{X_1}(t)=1-e^{-\lambda t} $. Z definicji $ X_n\ge 0 $.
\begin{gather*}
X_1\sim\text{Exp}\left(\lambda\right)
\end{gather*}
\end{proof}
\end{twr}