\chapter{14 grudnia 2015}
Komentarz związany z kolokwium\\
$ \left\{N(t)\right\}_{t\ge 0} $ jednorodny proces Poissona z intensywnością $ \lambda>0 $\\
$ cov\left(N(t),N(s)\right)=\lambda\left(t\wedge s\right)=\lambda\cdot\min\left\{s,t\right\} $\\
$ \mathbb E \left(N(t)\cdot N(s)\right)=cov\left(N(t),N(s)\right)+\mathbb E N(t)\mathbb E N(s)=\lambda\cdot\min\left\{s,t\right\}+\lambda s\cdot\lambda t $
dla $ t \ge s $\\
$ \mathbb E \left(N(t)\cdot N(s)\right)=\mathbb E \left(\left(N(t)\cdot N(s)\right)\cdot N(s)+N(t)^2\right) $
koniec komentarza

Niech $ \left(X_n\right)_{n\ge 0} $ będzie jednorodnym łańcuchem markowa z macierzą prawdopodobieństw przejść $ P=[p_{ij}] $. Oznaczmy
\begin{gather*}
\tau_j=\left \{
\begin{array}{l}
	\min\left\{n\ge 1:X_n(\omega)=j\right\}                     \\
	\infty , \text{ jeżeli }\forall_{n\ge 1}\; X_n(\omega)\neq j
\end{array}
\right .
\end{gather*}
moment Markowa.
\begin{gather*}
m_{jj}=\mathbb E _j\left(\tau_j\right)=\sum_{n=1}^{\infty }n\cdot f_{jj}^{[n]}=\int\limits_{\Omega}\tau_j\,dP\left(\cdot|X_0=j\right)
\end{gather*}
Przypomnienie
\begin{gather*}
f_{jj}^{[n]}=P\left(X_1\neq j,\dots,X_{n-1}\neq j,X_n=j|X_0=j\right)
\end{gather*}
\begin{defi}
Mówimy, że stan $ j\in C $ jest dodatnio powracający, jeżeli $ m_{jj}<\infty  $.\\
Mówimy, że stan $ j\in C $ jest 0 powracający, jeżeli $ m_{jj}=\infty  $.
\end{defi}
\textbf{Uwawga!}\\
Oczywiście, jeżeli $ j\in D=S\backslash C $ to $ P_j\left(\tau_j=\infty \right)>0 $. Zatem
\begin{gather*}
\forall_{j\in D}\;m_{jj}=\infty 
\end{gather*}
\begin{defi}
Mówimy, że miara probabilistyczna $ \mu  $na $ (S, 2^S) $ jest niezmiennicza (stacjonarna) dla $ j $ łańcucha Markowa $ \left(X_n\right)_{n\ge 0} $ o macierzy prawdopodobieństw przejść $ P $, jeżeli
\begin{gather*}
\forall_{j\in S}\;\sum_{i\in S}\mu_i\cdot p_{ij}=\mu_j\\
\underline{\mu}\circ P=\underline{\mu}
\end{gather*}
\end{defi}
\begin{gather*}
\mathcal P\left(S\right)\ni\underline{\mu}\to\underline{\mu}\circ P\in \mathcal P\left(S\right)\\
\left(\mu_0,\mu_1,\mu_2,\dots \right)\circ
\begin{bmatrix}
	p_{0,0} & p_{0,1} & \ldots \\
	p_{1,0} & p_{1,1} & \ldots \\
	\vdots  & \vdots  & \ddots
\end{bmatrix}=
\left(\mu_0,\mu_1,\mu_2,\dots \right)
\end{gather*}
\textbf{Uwaga!}\\
Niech $ X_0 $ ma rozkład $ \underline{\mu}\left[P\left(X_0=i\right)=\mu_i\right] $. Rozkład $ x_1 $?
\begin{gather*}
P\left(X_1=j\right)=
\sum_{j\in S}P\left(X_1=j|X_0=i\right)P\left(X_0=i\right)=
\sum_{i\in S}\mu_i\cdot p_{ij}=\underline{\mu}\circ P
\end{gather*}
\begin{align*}
\underline{\mu}&\text{ rozkład }X_0\\
\underline{\mu}\circ P&\text{ rozkład }X_1\\
\underline{\mu}\circ P^2&\text{ rozkład }X_2\\
\end{align*}
\begin{twr}
Niech $ \left(X_n\right)_{n\ge 0} $ będzie jednorodnym łańcuchem Markowa o macierzy prawdopodobieństw przejść $ P=\left[p_{ij}\right] _{S\times S}$. Jeżeli $ r\in C $ jest stanem dodatnio powracającym, wówczas istnieje dokładnie jedna miara probabilistyczna niezmiennicza $ \pi$ o własności $ \pi_r>0 $. Co więcej
\begin{gather*}
\pi_r=
\left \{
\begin{array}{ll}
\frac{1}{m_{rr}=\frac{1}{\mathbb E _r\tau_r}},&j=r\\
0,&j\notin [r]\\
\frac{1}{\mathbb E _r\tau_r}\cdot \left(1+P_{r_j}+\sum_{n=1}^{\infty }\sum_{i_1,\dots,i_n\in S\backslash\{r\}}
p_{ri_1}\cdot p_{i_1i_2}\cdot\dots p_{i_{n-1}j}\right),&j\in [r]\wedge j\neq r
\end{array}
\right .
\end{gather*}
\end{twr}
\textbf{Wniosek}
\begin{gather*}
C=C_+\cup C_0\\
C_+=\left\{r\in C:\mathbb E _r\tau_r<\infty \right\}=\left\{j\in C:\mu_j>0\text{ dla pewnej }\underline{\mu}\circ P=\underline{\mu}\right\}
\end{gather*}
\begin{twr}
Niech $ \left(X_n\right)_{n\ge 0} $ będzie jednorodnym łańcuchem Markowa na przestrzeni stanów $ S $ o macierzy prawdopodobieństw przejść $ P[p_{ij}] $. Łańcuch $ \left(X_n\right)_{n\ge 0} $jest stacjonarny wtedy i tylko wtedy, gdy $ \mu^{(0)}=\mathcal L\left(X_0\right) $ rozkład początkowy jest miarą niezmienniczą.
\end{twr}
\begin{proof}$ \Rightarrow $
Ze stacjonarności $ \left(X_0,X_1,\dots,X_n\right)\sim \left(X_1,X_2,\dots,X_{n+1}\right)$. $ \mathcal L(X_0)=\mathcal L(X_1) $. Ale pamiętamy, że $ \underline{\mu}^{(1)}=\underline{\mu}^{(0)}\circ P $, czyli mamy $ \underline{\mu}^{(0)}=\mu^{(0)}\circ P $.
\textbf{Uwaga!}\\
Iterując $ \underline{\mu}^{(n)}=\underline{\mu}^{(0)}\circ P^n $ dla dowolnego $ n\ge 0 $\\
$ \Leftarrow $\\
Załóżmy, że $ \underline{\mu}^{(0)}=\underline{\mu}^{(0)}\circ P $.\\
$ \mu_{(X_0,\dots,X_n)}=\mu_{(X_k,\dots,X_{n+k})} $ dla dowolnego $ k=1,2,\dots $?
\begin{gather*}
P\left(X_0=i_0,X_1=i_1,\dots,X_n=i_n\right)=
P\left(X_k=i_0,X_{k+1}=i_1,\dots,X_{k+n}=i_n\right)
\end{gather*}
dla dowolnego wyboru $ i_0,i_1,\dots,i_n\in S $
\begin{align*}
&P\left(X_0=i_0\right)p_{i_0i_1}\cdot p_{i_1i_2}\cdot\ldots \cdot p_{i_{n-1}i_n}\\
&P\left(X_k=i_0\right)p_{i_0i_1}\cdot p_{i_1i_2}\cdot\ldots \cdot p_{i_{n-1}i_n}\\
&P\left(X_k=i_0\right)=\left(\underline{\mu}^{(0)}\circ P^k\right)_{i_0}=\mu_{i_0}^{(0)}=P\left(X_0=i_0\right)
\end{align*}
Stąd $ P\left(X_0=i_0\right)p_{i_0i_1}\cdot p_{i_1i_2}\cdot\ldots \cdot p_{i_{n-1}i_n}=P\left(X_k=i_0\right)p_{i_0i_1}\cdot p_{i_1i_2}\cdot\ldots \cdot p_{i_{n-1}i_n} $ dla dowolnego $ k\ge 0 $. Czyli otrzymaliśmy stacjonarność $ X_0,X_1,\dots  $
\end{proof}
\begin{twr}
Niech $ \left(X_n\right)_{n\ge 0} $ będzie jednorodnym łańcuchem Markowa z macierzą prawdopodobieństw przejść $ P=[p_{ij}] $. Wówczas
\begin{enumerate}
\item
\begin{gather*}
\lim\limits_{n\to\infty} \frac{1}{n}\sum_{k=1}^{n}p_{jj}^{(k)}=\frac{1}{m_{jj}}
\end{gather*}
\item 
\begin{gather*}
\lim\limits_{n\to\infty} p_{jj}^{(n)}=\frac{1}{m_{jj}}
\end{gather*}
jeśli dodatkowo założymy, że $ j $ jest aperiodyczny
\item 
\begin{gather*}
\lim\limits_{n\to\infty} p_{jj}^{(n)}=\frac{d}{m_{jj}}
\end{gather*}
jeżeli dodatkowo założymy, że $ d(j)=d $.
\end{enumerate}
\end{twr}
\textbf{Uwaga!}\\
Powyższe twierdzenie nosi nazwę średniego twierdzenia ergodycznego na $ l^1 $.
\newpage
\begin{center}
{\Large\textsc{Proces ruchu Browna}}
\end{center}
Intuicyjne wprowadzenie w teorię procesów ruchu Browna (procesów Wienera).\\
$ X_1,\dots  $ ciąg niezależnych zmiennych losowych o tym samym rozkładzie takich, że 
$ P\left(X_k=-1\right)=P\left(X_k=1\right)=\frac{1}{2} $\\
$ X_t^{\substack{\Delta x\\\Delta t}} =\Delta x\cdot X_1+\Delta x\cdot X_2+\dots\Delta x\cdot X_{\left \lfloor\frac{t}{\Delta t}\right \rfloor}$ dla $ t\ge 0 $\\
Spacer losowy o skokach $ \pm\Delta x $\\
\begin{gather*}
\mathbb E X_t^{\substack{\Delta x\\\Delta t}}=\sum_{j=1}^{\left \lfloor\frac{t}{\Delta t}\right \rfloor}\mathbb E \Delta x\cdot X_1=0\\
\Var X_t^{\substack{\Delta x\\\Delta t}}=\sum_{j=1}^{\left \lfloor\frac{t}{\Delta t}\right \rfloor}\Var\left(\Delta x\cdot X_j\right)=\Delta x^2\cdot \left \lfloor\frac{t}{\Delta t}\right \rfloor\cdot 1
\end{gather*}
Dobieramy $ \Delta x,\Delta t $