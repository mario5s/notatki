\chapter{7 grudnia 2015}
\begin{defi}
Niech $ \left(X_n\right)_{n\ge 0} $ będzie łańcuchem Markowa (z czasem dyskretnym i skończoną lub przeliczalną przestrzenią stanów $ S $). Mówimy, że łańcuch $ X\left(_n\right)_{n\ge 0} $ jest nieprzewiedlny, gdy
\begin{gather*}
\forall_{i,j\in S}\;i\longleftrightarrow j.
\end{gather*}
\end{defi}
\begin{twr}
Niech $ \left(X_n\right)_{n\ge0} $ będzie jednorodnym łańcuchem Markowa(z czasem dyskretnym i skończoną lub przeliczalną przestrzenią stanów $ S $) o macierzy prawdopodobieństw przejść $ \left[p_{ij}\right]_{S\times S} $. Jeżeli $ i\longleftrightarrow j $, to $ d(i)=d(j) $.\\
Przypomnienie: $ d(i)=NWD\left\{n\ge0:p_{ii}^{(n)}>0\right\} $
\end{twr}
\begin{proof}
$ i\longleftrightarrow j,P_{ij}^{(n)}>0,p_{ji}^{(m)}>0 $ dla pewnych $ n.m $ (myślimy, że $ i\neq j $)
\begin{gather*}
p_{ii}^{(n+m)}=\sum_{k\in S}p_{ik}^{(n)}p_{kj}^{(m)}\ge p_{ij}^{(m)}>0
\end{gather*}
$ p_{ii}^{l(n+m)}>0 $
\begin{gather*}
\underbrace{p_{ii}^{(n+m)}\cdot p_{ii}^{(n+m)}\dots p_{ii}^{(n+m)}}_{l\text{ razy}}
\end{gather*}
$ d(j)=p_{jj}^{(s)}>0 $
\begin{gather*}
p_{ii}^{(n+s+m)}\ge p_{ij}^{(n)}p_{jj}^{(s)}p_{ji}^{(m)}
\end{gather*}
$ p_{ii}^{(n+m)}>0 \\
d(i)|n+s+m=n+m+s\\
d9i)|n+m$\\
Stąd $ d(i)|s $ każde takie $ s $, ze $ p_{jj}^{(s)}>0 $.\\
Z teorii podzielności liczb naturalnych $ d(i)|d(j) $.\\
Z symetrii $ \longleftrightarrow\; d(j)|d(i) $.\\
Ostatecznie $ d(i)=d(j) $
\end{proof}
\textbf{Winosek}\\
Jeżeli łańcuch Markowa jest nieprzewiedlny, to
\begin{gather*}
\forall_{i,j\in S}\; d(i)=d(j)
\end{gather*}
Oznaczenia:\\
$ \left(X_n\right)_{n\ge0} $ - łańcuch Markowa jednorodny o macierzy prawdopodobieństw przejść $ P=\left[p_{ij}\right] _{S\times S}$
\begin{align*}
&f_{ij}^{(n)}=P\left(X_1\neq j,\dots,X_{n-1}\neq j,X_n=j\right),\quad n\ge 1\\
&f_{ij}=\sum_{n=1}^{\infty }f_{ij}^{(n)}=
\sum_{n=1}^{\infty }P\left(X_1\neq j,\dots,X_{n-1}\neq j,X_n=j\right)
=\\=
&P\left(\bigcup_{n=1}^{\infty }\left\{X_1\neq j,\dots,X_{n-1}\neq j,X_n=j\right\}\right)
=\\=
&P\left(\exists_{n\ge 1}\;X_n=j|X_0=i\right)
\end{align*}
Moment Markowa
\begin{gather*}
T_{ij}\stackrel{df}{=}\inf\left\{n\ge1:X_n=j,X_0=i \right\}
\end{gather*}
Pierwsza chwila uderzenia w stan $ j $.
\begin{gather*}
T_j\stackrel{df}{=}\inf\left\{n\ge1:X_n=j\right\}
\end{gather*}
Ogólnie
\begin{align*}
&T_D\stackrel{df}{=}\inf \left\{n\ge1:X_n\in D\right\},D\subseteq S\qquad\text{hitting time}\\
&P\left(T_{ij}<\infty |X_0=i\right)=P_i\left(T_{ij}<\infty \right)=f_{ij}
\end{align*}
\begin{defi}
Niech $ \left(X_n\right)_{n\ge 1} $ będzie łańcuchem Markowa z macierzą prawdopodobieństw przejść $ P\left[p_{ij}\right]_{S\times S} $. Mówimy, że stan $ j\in S $ jest powracający, jeżeli $ f_{jj}=1 $.\\
Mówimy, że stan $ j\in S $ jest chwilowy, jeżeli $ f_{jj}<1 $.
\end{defi}
\begin{twr}[Charakteryzacja stanów powracających]Neich $ \left(X_n\right)_{n\ge0} $ będzie jednorodnym łańcuchem markowa z macierzą prawdopodobieństw przejść $ P=\left[p_{ij}\right] _{S\times S}$. Wówczas następujące warunki są równoważne:
\begin{enumerate}
\item stan $ j $ jest powracający
\item $ P\left(\left\{\omega\in\Omega:X_n(\omega)=j\text{ dla nieskończenie wielu }n\in \mathbb N \right\}|X_0=j\right)=1 $
\item $ \sum_{n=1}^{}p_{jj}^{(n)}=\infty $
\end{enumerate}
\end{twr}
\begin{align*}
&L(\omega)=\#\left\{n\ge 1:X_n=j\right\}=\sum_{n=1}^{\infty }\mathbbm1_{\left\{j\right\}}\left(X_n(\omega)\right)\\
&\left\{L_j\ge1\right\}=\left\{T_j(\omega)<\infty \right\}\\
&\left\{L_j\ge 2\right\}=\left\{\exists_{n\ge 1}\;X_{t_j+n}=j\right\}\\
&T_j^{[2]}(\omega)=\inf\left\{n>T_j(\omega):X_n(\omega)=j\right\}&&\text{chwila drugiego trafienia w }j\\
&T_j^{[k]}(\omega)=\inf\left\{n>T_j^{[k-1]}(\omega):X_n(\omega)=j\right\}&&\text{chwila $ k $-tego trafienia do }j
\end{align*}
$ f_{jj}=
P\left(T_j<\infty \right)=
P\left(L_j\ge1\right)\\
P\left(L_j\ge2\right)=P\left(\exists_{n\ge 1}\;X_{t_j^{[1]}=j|X_0=j}\right) $
\begin{align*}
T_j^{[1]}=&\sum_{l=1}^{\infty }P\left(\exists_{n\ge 1}\;X_{T_j^{[1]}+n}(\omega)=j|T_j^{[1]}=l\right)P\left(T_j^{[1]}=l|X_0=j\right)
=\\=&
\sum_{l=1}^{\infty }f_{jj}\cdot P\left(T_j^{[1]}=l|X_0=j\right)
=\\=&
f_{jj}\cdot\sum_{l=1}^{\infty } P\left(T_j^{[1]}=l|X_0=j\right)
=\\=&
f_{jj}\cdot P\left(T_j^{[1]}<\infty |X_0=j\right)
=
f_{jj}^2
\end{align*}
W ten sam sposób
\begin{gather*}
P\left(L_j\ge m\right)=f_{jj}^m
\end{gather*}
\begin{proof}
$ (1)\Rightarrow(2) $\\
Jeżeli $ f_{jj}=1 $, to
\begin{align*}
&\forall_{n\in \mathbb N }\;P\left(L_j\ge m\right)=f_{jj}^{m}=1^m=1\\
&P\left(\left\{\omega\in\Omega:L_j(\omega)=\infty \right\}\right)=P\left(\bigcap_{n=1}{\infty }\left\{L_j\ge m\right\}\right)=\lim\limits_{m\to\infty} P\left(L_j\ge m\right)=1\\
&P\left(\left\{\omega\in\Omega:\sum_{n=1}^{\infty }\mathbbm1_{\left\{j\right\}}\left(X_n(\omega)=\infty )\right)\right\}\right)=P\left(\left\{\omega\in\Omega:L_j(\omega)=\infty \right\}\right)=1
\end{align*}
$ (2)\Rightarrow(3) $
\begin{align*}
&\sum_{n=1}^{\infty }p_{jj}^{(n)}
=\\=&
\sum_{n=1}^{\infty }\int\limits_{\Omega}\mathbbm1_{\left\{j\right\}}\left(X_n\right)\,dP_j
=\\=&
\sum_{n=1}^{\infty }\mathbb E _j\mathbbm1_{\left\{j\right\}}\left(X_n\right)
=\\=&
\mathbb E _j\sum_{n=1}^{\infty }\mathbbm1_{\left\{j\right\}}\left(X_n\right)
=\\=&
\mathbb E _jL_j(\omega)=\infty 
\end{align*}
$ (3)\Rightarrow(1) $\\
Przypuśćmy, że $ f_{jj}<1  $.\\
$ P_j\left(L_j\ge m\right)=f_{jj}^m $
\begin{gather*}
\sum_{n=1}^{\infty }P_j\left(L_j\ge m\right)=\sum_{n=1}^{\infty }f_{jj}^m=\frac{f_{jj}}{1-f_{jj}}<\infty \\
\mathbb E _{P_j}\left(L_j\right)=\sum_{n=1}^{\infty }p_{jj}^{(n)}=\infty 
\end{gather*}
Sprzeczność.
\end{proof}
\begin{lem}
$ \left(X_n\right)_{n\ge 0} $ jednorodny łańcuch Markowa z macierzą prawdopodobieństw przejść $ P=\left[p_{ij}\right] _{S\times S}$. Jeżeli $ j $ jest stanem powracającym oraz $ j\longleftrightarrow i (f_{ji}>0)$. Wówczas $ f_{ij}=1 $.
\end{lem}
\begin{proof}$  $\\
$ j \longrightarrow i \Rightarrow \exists_{n_0}\;p_{ji}^{(n_0)}>0$\\
$ P\left(L_j=\infty |X_0=j\right)=1 $. Zatem istnieje $ n>n_0 $ takie, że $ X_n=j $.\\
$ P\left(\left\{\omega\in\Omega:X_n(\omega)=j\text{ dla pewnego }n>n_0|X_0=j\right\}\right)=1 $
\begin{align*}
1=&P_j\left(\exists_{n>n_0}\;X_n=j\right)
=\\=&
\sum_{l\in S}P\left(\exists_{n>n_0}\;X_n=j|X_{n_0}=l\right)P_j\left(X_{n_0}=l\right)
=\\=&
\sum_{l\in S}P_j\left(\exists_{n\ge 1}\;X_{n+n_0}=j|X_{n_0}=l\right)p_{jl}^{(n_0)}
=\\=&
\sum_{l\in S}f_{ij}p_{jl}^{(n_0)}
\end{align*}
\begin{align*}
&1-\sum_{l\in S}f_{ij}p_{jl}^{(n_0)}
=\\=&
\sum_{l\in S}p_{jl}^{(n_0)}-\sum_{l\in S}f_{ij}p_{jl}^{(n_0)}
=\\=&
\sum_{l\in S}\underbrace{p_{jl}^{(n_0)}}_{\ge 0}(\underbrace{1-f_{lj}}_{\ge 0})
\end{align*}
Jeżeli $ p_{jl}^{(n_0)}>0 $, to $ 1-f_{lj}=0 $\\
Właśnie $ j\longrightarrow i $ spełnia $ p_{ji}^{(n_0)}>0 $ zatem ostatecznie $ f_{ij}=1 $
\end{proof}
\begin{lem}
Niech $ \left(X_n\right)_{n\ge 0} $ będzie jednorodnym łańcuchem Markowa o macierzy prawdopodobieństw przejść $ P=\left[p_{ij}\right] $. Jeżeli $ j\in S $ jest stanem chwilowym to dla dowolnego $ i\in S $
\begin{gather*}
\sum_{n=1}^{\infty }p_{ij}^{(n)}<\infty 
\end{gather*}
co implikuje
\begin{gather*}
\lim\limits_{n\to\infty} p_{ij}^{(n)}=0
\end{gather*}
\end{lem}
\begin{proof}
Ustalmy $ i\in S $. Jeżeli $ \forall_{n\ge 1}\;p_{ij}^{(n)}=0 $ (tzn. $ \neg\left(i\longleftrightarrow j\right) $), to oczywiście
\begin{gather*}
\sum_{n=1}^{\infty }p_{ij}^{(n)}=\sum_{n=1}^{\infty }0=0
\end{gather*}
Zatem będziemy rozpatrywali przypadek $ \exists_{n\ge 1}\;p_{ij}^{(n)}>0 $.
\begin{align*}
p_{ij}^{(n)}
=&
P\left(X_n=j|X_0=i\right)
=\\=&
\frac{P\left(X_n=j,X_0=i\right)}{P\left(X_0=i\right)}
=\\=&
\frac{P\left(\left\{X_n=j\right\}\cap\left\{T_j\le n\right\}\cap \left\{X_0=i\right\}\right)}{P\left(X_0=i\right)}
=\\=&
\frac{P\left(\left\{X_n=j\right\}\cap\bigcup\limits_{k=1}^\infty \left\{T_j=k\right\}\cap \left\{X_0=i\right\}\right)}{P\left(X_0=i\right)}
=\\=&
\sum_{k=1}^{n }\frac{P\left(\left\{X_n=j\right\}\cap \left\{X_1\neq j,\dots,X_{k-1}\neq j,X_k=j\right\}\cap \left\{X_0=i\right\}\right)}{P\left(X_0=i\right)}
=\\=&
\sum_{k=1}^{n}\frac{P\left(X_n=j|X_k=j,X_{k-1}\neq j,\dots,X_0=i\right)}{P\left(X_0=i\right)}P\left(X_k=j,X_{k-1}\neq j,\dots,X_0=i\right)
=\\=&
\sum_{k=1}^{n}p_{jj}^{(n-k)}f_{ij}^{[k]}=p_{ij}^{(n)}
\end{align*}
W ostatnim przejściu pamiętamy, że $ \sum_{n=0}^{\infty }p_{jj}^{(n)}<\infty  $

\begin{align*}
\sum_{n=1}^{\infty }p_{dół}^{(n)}
=&
\sum_{n=1}^{\infty }\left(\sum_{k=1}^{n}f_{ij}^{[k]}p_{jj}^{(n-k)}\right)
=\\=&
\sum_{k=1}^{\infty }f_{ij}^{[k]}\sum_{n=k}^{n}p_{jj}^{(n-k)}
=\\=&
\left(\sum_{n=1}^{n}p_{jj}^{(n)}\right)\sum_{k=1}^{\infty }f_{ij}^{[k]}<\infty 
\end{align*}
\begin{gather*}
\sum_{n=1}^{\infty }p_{ij}^{(n)}<\infty 
\end{gather*}
\end{proof}
\textbf{Wniosek}\\
Jeżeli przestrzeń stanów $ S $ dla jednorodnego łańcucha Markowa $ \left(X_n\right)_{n\ge 0} $ jest skończona ($ \#S<\infty  $), to istnieje co najmniej jedne stan powracający.
\begin{proof}
Przypuśćmy, że wszystkie stany są chwilowe.
\begin{gather*}
\forall_{i\in S}\forall_{j\in S}\;\sum_{n=0}^{\infty }p_{dół}^{(n)}=\rho_{ij}<\infty 
\end{gather*}
\begin{gather*}
\sum_{j\in S}\sum_{i\in S}\sum_{n=0}^{\infty }p_{dół}^{(n)}=\sum_{i,j\in S}\rho_{ij}<\infty \\
\sum_{n=0}^{\infty }\sum_{i\in S}\sum_{j\in S}p_{dół}^{(n)}=\sum_{n=0}^{\infty }\left(\#S\cdot 1\right)=\infty 
\end{gather*}
\end{proof}
\textbf{Uwaga!}\\
Może się zdarzyć, że dla $ \#S=\infty  $ nie ma w ogóle stanów powracających.

Oznaczenia\\
Część konserwatywna procesu $ \left(X_n\right)_{n\ge 0} $
\begin{gather*}
C=\left\{j\in S:j\text{ jest stanem powracającym}\right\}=\left\{j\in S:\sum_{n=1}^{\infty }p_{jj}^{(n)}=\infty \right\}
\end{gather*}
Część dysypatywna procesu $ \left(X_n\right)_{n\ge 0} $
\begin{gather*}
D=\left\{j\in S:p_{jj}^{(n)}\infty \right\}
\end{gather*}
\begin{defi}
Niech $ \left(X_n\right)_{n\ge 0} $ będzie jednorodnym łańcuchem Markowa na przestrzeni stanów $ S $ o macierzy prawdopodobieństw przejść $ P=\left[p_{ij}\right] _{S\times S}$. Mówimy, że $ A\subseteq S $ jest zamknięty (niezmienniczy), jeśli spełnia
\begin{gather*}
\forall_{i\in A}\forall_{j\notin A}\;p_{ij}=0
\end{gather*}
Inaczej
\begin{gather*}
P\mathbbm1_{A}(i)=\sum_{j\in A}p_{ij}=\sum_{j\in A}\mathbbm1_{A}(j)p_{ij}
\end{gather*}
\end{defi}
\begin{twr}
Część konserwatywna procesu jest zbiorem zamkniętym.
\begin{proof}
Niech $ i\in C $ oraz $ i\longleftrightarrow j $. Przypuśćmy $ p_{ij}^{(n_0)}>0 $. Wtedy $ f_{ij}=1 $ z lematu.\begin{gather*}
\exists_{n_0\ge 1}\;p_{ij}^{(n_0)}>0
\end{gather*}
\begin{align*}
\sum_{k=1}^{\infty }p_{jj}^{(k)}
\ge\\\ge&
\sum_{k=m_0+n_0+1}^\infty p_{jj}^{(k)}
\ge\\\ge&
\sum_{k=m_0+n_0+1}^\infty p_{ji}^{(m_0)}p_{ii}^{(k-n_0-m_0)}p_{ij}^{(n_0)}
=\\=&
\underbrace{p_{ji}^{(m_0)}}_{\ge 0}
\sum_{l=1}^\infty
\underbrace{p_{ii}^{(l)}}_{\infty }
\underbrace{p_{ij}^{(n_0)}}_{\ge 0}
=\infty 
\end{align*}
\end{proof}
\end{twr}