\chapter{28 września 2015}
Procesy stochastyczne - teoria służąca do opisu analizy (wnioskowań) zjawisk losowych ewoluujących w czasie. Wywodzą się z rachunku prawdopodobieństwa. Modelowanie rzeczywistości obarczonej niepewnością (losowością). Przeciwieństwo równań różniczkowych.
\begin{gather*}
y'=f(y,t)\\
y(t)=g(t,y_0)
\end{gather*}
Równania różniczkowe dostarczają modeli deterministycznych. Idealizuje, możliwe w warunkach laboratoryjnych. Bardzo dużo praktycznych zagadnień nie ma charakteru deterministycznego (a jak już ma to będzie to determinizm chaotyczny). Chaos to nie to samo, co losowość. Losowość tkwi w głębi natury. Dla przykładu mechanika kwantowa.

Podstawowe elementy (pojęcia) procesów stochastycznych:\\
Przestrzeń probabilistyczna $(\Omega,\mathcal F,P)$\\
$ \left .
\begin{array}{l}
 \Omega \text{ - zbiór zdarzeń elementarnych}\\
 \omega\in\Omega \text{ - zdarzenie elementarne}\\
\end{array}
\right \} $ pojęcia pierwotne\\
$ \mathcal F $ - $ \sigma $-ciało podzbioru $ \Omega $

\newpage
\begin{defi}[$ \sigma$-ciało]
Mówimy, że rodzina $ \mathcal F  $ podzbiorów $ \Omega $ jest $ \sigma $-ciałem, jeśli spełnia:
\begin{enumerate}
\item
\begin{gather*}
\emptyset\in \mathcal F ,\quad\Omega\in \mathcal F,
\end{gather*}
\item
\begin{gather*}
\forall_{A\in\Omega}\left[A\in \mathcal F \Rightarrow A^\mathrm c=\Omega\backslash A\in \mathcal F \right],
\end{gather*}
\item 
\begin{gather*}
\forall_{ A_1,A_2,\dots\in\Omega}\left[\forall_n A_n\in \mathcal F \Rightarrow\bigcup_{n=1}^\infty A_n\in \mathcal F \right],
\end{gather*}
\end{enumerate}
\end{defi}
\begin{defi}
Mówimy, że funkcja zbioru $ P $ określona na $ (\Omega,\mathcal F ) $ jest $ \sigma $-addytywnym prawdopodobieństwem (miarą probabilistyczną $ \sigma $-addytywną), jeżeli spełnia:\begin{enumerate}
\item 
\begin{gather*}
P:\mathcal F \rightarrow[0,1],
\end{gather*}
\item 
\begin{gather*}
P(\emptyset)=0,
\end{gather*}
\item 
\begin{gather*}
P(\Omega)=1,
\end{gather*}
\item warunek $ \sigma $-addytywności
\begin{gather*}
\forall_{A_1,A_2,\ldots\in \mathcal F }
\left[\forall_{i\neq j}A_i\cap A_j=\emptyset\Rightarrow P\left(\bigcup_{n=1}^\infty A_n\right)=\sum_{n=1}^{\infty }P(A_n)\right]
\end{gather*}
\end{enumerate}
\end{defi}
\begin{defi}[Proces stochastyczny]
Procesem stochastycznym na przestrzeni probabilistycznej $(\Omega,\mathcal F,P)$ nazywamy rodzinę zmiennych losowych $ \{X_t\}_{t\in T} $ określonych na $(\Omega,\mathcal F,P)$.\\
$ T $ - zbiór indeksów (czasowych, gdy $ T $ interpretujemy jako czas)
\end{defi}
\begin{defi}[Zmienna losowa]
Zmienna losowa $ X_t $.
\begin{gather*}
\forall_{t\in T}\forall_{B\in \mathfrak B_\mathbb R}\,
X_t^{-1}(B)\in \mathcal F 
\end{gather*}
$ X_t $ jest $ \mathcal F  $ mierzalne; $ X_t $ jest $ \sigma (\mathcal F ,\mathfrak B_\mathbb R) mierzalna $
\begin{itemize}
\item $ \mathfrak B_\mathbb R $ - $ \sigma $-ciało zbiorów borelowskich w $ \mathbb R  $
\item $ \mathfrak B_\mathbb R=\sigma\left(\left\{\left(\alpha,\beta\right):\alpha<\beta\right\}\right) $
\end{itemize}
\end{defi}
\textbf{Uwaga!}\\
\begin{gather*}
\text{card}(\mathfrak B_\mathbb R)=\mathfrak c<2^\mathfrak c=\text{card}\left(\mathcal L_\mathbb R \right)
\end{gather*}
\begin{defi}
Niech $ \Omega $ będzie ustalonym zbiorem. Mówimy, że rodzina $ \mathcal C $ podzbiorów $ \Omega $ tworzy ciało, jeżeli spełnia:
\begin{enumerate}
\item 
\begin{gather*}
\emptyset,\,\Omega\in\mathcal C
\end{gather*}
\item 
\begin{gather*}
\forall_{A\subseteq\Omega}
\left[A\in\mathcal C\Rightarrow A^\mathrm c\in\mathcal C\right]
\end{gather*}
\item 
\begin{gather*}
\forall_{A,B\in\Omega}
\left[A<B\in\mathcal C\Rightarrow A\cup B\in\mathcal C\right]
\equiv\\
\equiv
\left[A_1,A_2,\dots,A_n\in\mathcal C,\quad n\in \mathbb{N}\Rightarrow \bigcup_{j=1}^n A_j\in \mathcal C\right]
\end{gather*}
\end{enumerate}
\end{defi}
\begin{defi}
Mówimy, że funkcja zbioru $ \mu $ określona na $ (\Omega,\mathcal C ) $ jest miarą probabilistyczną $ \sigma
$-addytywną, jeśli spełnia:
\begin{enumerate}
\item 
\begin{gather*}
\mu(\emptyset)=0\\
\mu(\Omega)=1\\
\forall_{A\in \mathcal C }\,
0\le\mu(A)\le1
\end{gather*}
\item 
\begin{gather*}
\forall_{A_1,A_2,\ldots\in \mathcal C }
\left[\forall_{i\neq j}A_i\cap A_j=\emptyset\wedge \bigcup_{n=1}^\infty A_n \in \mathcal C  \Rightarrow \mu\left(\bigcup_{n=1}^\infty A_n\right)=\sum_{n=1}^{\infty }\mu\left(A_n\right)\right]
\end{gather*}
\end{enumerate}
\end{defi}